\chapter{Revisão Bibliográfica}
\section{Processamento de Imagens}
Nesta seção, serão abordados os conceitos fundamentais de processamento de imagens, incluindo técnicas e algoritmos utilizados para a análise e manipulação de imagens digitais. Também será discutida a importância do processamento de imagens no SEP para a detecção e classificação de falhas em equipamentos de linhas de transmissão de energia elétrica.

\section{Datasets}
Será discutida a influência dos datasets na performance dos modelos de processamento de imagem, considerando a importância da escolha de conjuntos de dados representativos e diversificados para o treinamento e avaliação dos modelos.

\section{Redes Neurais}
Aqui, serão discutidos os diferentes tipos de redes neurais aplicáveis ao processamento de imagens, com foco em suas arquiteturas e aplicações específicas para avaliação de desempenho.

\subsection{Métricas}
Serão apresentadas as principais métricas utilizadas para avaliar a eficácia dos processamentos de imagem, como acurácia, tempo de processamento, precisão, recall, entre outras.

\section{Influência de Datasets na Performance dos Modelos}
Esta seção tratará da importância dos datasets na performance dos modelos de processamento de imagem, incluindo a análise de diferentes conjuntos de dados e suas características.

\section{Métodos de Ajuste de Parâmetros e Combinação de Processamentos}
Serão explorados os métodos para ajuste automático de parâmetros e a combinação de diferentes técnicas de processamento de imagem para otimização dos resultados.
