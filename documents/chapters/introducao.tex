\chapter{Introdução}

\section{Proposta}

O presente estudo tem como objetivo desenvolver uma metodologia capaz de comparar, selecionar, combinar e aprimorar técnicas de processamento de imagem para a detecção e classificação de falhas em cadeias de isoladores. Para isso, serão estabelecidas métricas para avaliar a eficácia dos processamentos de imagem, considerando aspectos como acurácia e tempo de processamento. Além disso, serão construídos modelos de redes neurais para avaliar o desempenho dos processamentos, podendo abranger tarefas como classificação, detecção e regressão. No decorrer do estudo, serão construídos modelos de redes neurais voltados para a avaliação do desempenho das técnicas de processamento de imagem, sem a intenção de definir um modelo ideal.

Também será analisado o impacto da escolha do modelo de rede neural no desempenho do processamento, visto que diferentes modelos podem gerar resultados distintos para um mesmo processamento. A influência do dataset na eficácia do processamento será outro aspecto a ser investigado, considerando possíveis variações nos resultados devido ao uso de diferentes conjuntos de dados. Para aprimorar os processamentos de imagem, será desenvolvida uma metodologia que permita a combinação de diferentes processamentos unitários (processamentos de imagem que realizam uma única operação). Além disso, será criado um método de ajuste automático de parâmetros das técnicas de processamento de imagem, com o intuito de otimizar seus resultados sem exigir extensa intervenção manual.

A metodologia proposta será desenvolvida dentro de um conjunto de restrições previamente estabelecidas, garantindo um escopo bem delimitado e viável dentro do período de realização da dissertação. Primeiramente, o estudo será restrito à detecção e classificação de falhas em cadeias de isoladores elétricos, não abrangendo outros componentes elétricos. O uso de imagens previamente adquiridas será uma diretriz, de modo que apenas imagens já disponíveis ou capturadas por métodos convencionais serão utilizadas, sem o desenvolvimento de novas técnicas de aquisição de imagens. Além disso, a metodologia será aplicada exclusivamente a técnicas de processamento de imagem já conhecidas, sem a criação de novos algoritmos de base.

Os modelos de redes neurais desenvolvidos terão o propósito único de avaliar o impacto das redes sobre os processamentos de imagem, sem a intenção de definir um modelo definitivo para diagnóstico industrial. A análise será conduzida utilizando conjuntos de dados já existentes ou obtidos por métodos convencionais, sem a necessidade de criar um novo dataset específico para o estudo. A otimização contemplada estará limitada ao ajuste de parâmetros das técnicas existentes, não incluindo o desenvolvimento de novas abordagens baseadas em inteligência artificial para otimização dos processamentos. Por fim, toda a avaliação será realizada em ambiente controlado, sem a realização de testes em ambientes industriais reais.

A Figura \ref{fig:proposta} ilustra o diagrama da proposta de metodologia.

\begin{figure}[h]
    \label{fig:proposta}
    \centering
    \caption{Diagrama da proposta de metodologia}
    \includegraphics[width=\textwidth]{img/proposta.png}
    \fonte{Autor.}
\end{figure}

\section{Objetivo geral}

O objetivo geral deste estudo é desenvolver uma metodologia capaz de comparar, selecionar, combinar e aprimorar técnicas de processamento de imagem para a detecção e classificação de falhas em cadeias de isoladores.

\section{Objetivos específicos}

Para alcançar esse objetivo, foram definidos os seguintes objetivos específicos:

\begin{itemize}
    \item Estabelecer métricas para avaliar a eficácia dos processamentos de imagem, considerando aspectos como acurácia e tempo de processamento.
    \item Determinar o tipo de modelo de redes neurais ideal para avaliar o desempenho dos processamentos, podendo abranger classificação, detecção e regressão.
    \item Construir modelos de redes neurais destinados à avaliação do desempenho das técnicas de processamento de imagem, sem o intuito de encontrar um modelo definitivo.
    \item Analisar o impacto da escolha do modelo de rede neural no desempenho do processamento, considerando que diferentes modelos podem gerar diferentes resultados para um mesmo processamento.
    \item Avaliar a influência do dataset na eficácia do processamento, considerando possíveis variações nos resultados devido à utilização de diferentes conjuntos de dados.
    \item Desenvolver uma metodologia para o aprimoramento dos processamentos de imagem por meio da combinação de diferentes abordagens unitárias.
    \item Criar um método de ajuste automático de parâmetros dos processamentos de imagem, visando otimizar seus resultados sem a necessidade de intervenção manual extensa.
\end{itemize}

\section{Justificativa}

A crescente demanda por sistemas automatizados de inspeção de cadeias de isoladores evidencia a necessidade de técnicas avançadas de processamento de imagem para a detecção precoce de falhas. Conforme demonstrado por Gonzalez e Woods \cite{Gonzalez2008}, a análise digital de imagens permite extrair características relevantes para identificar anomalias em componentes elétricos, possibilitando diagnósticos mais precisos. Ademais, o emprego de redes neurais tem se destacado na resolução de problemas complexos de classificação e detecção, conforme ressaltado por LeCun et al. \cite{LeCun2015} e Krizhevsky et al. \cite{Krizhevsky2012}, contribuindo para a robustez dos sistemas de inspeção.

Estudos recentes apontam que a combinação de diferentes técnicas de processamento de imagem, aliada ao ajuste automático de parâmetros, pode resultar em melhorias significativas no desempenho dos sistemas de diagnóstico \cite{Li2019}. Assim, a proposta deste trabalho visa desenvolver uma metodologia que integre esses avanços, buscando não apenas aprimorar a acurácia e a eficiência dos processamentos, mas também possibilitar uma análise comparativa que leve em conta a influência de diferentes modelos e datasets.

Dessa forma, esta dissertação justifica-se pela necessidade de inovar na abordagem de detecção e classificação de falhas em cadeias de isoladores, promovendo ganhos práticos para a segurança e manutenção das redes elétricas, e contribuindo para a evolução do estado da arte em processamento de imagem e aprendizado de máquina.

\section{Cronograma}

% A seguir, apresenta-se a estrutura das etapas e subetapas que compõem o desenvolvimento deste trabalho. A lista a seguir organiza os principais tópicos a serem abordados ao longo da pesquisa, detalhando desde a introdução até a conclusão, incluindo as metodologias, a revisão bibliográfica, a implementação dos modelos e a análise dos resultados.

% \begin{itemize}
%     \item \textbf{1. Introdução}
%     \begin{itemize}
%         \item Justificativa com apresentação do problema e relevância
%         \item Objetivos gerais e específicos
%         \item Metodologia geral adotada
%         \item Estrutura da dissertação
%     \end{itemize}
    
%     \item \textbf{2. Revisão Bibliográfica}
%     \begin{itemize}
%         \item Processamento de imagens
%         \item Redes neurais para avaliação de processamentos de imagem
%         \item Métricas para análise de desempenho (acurácia, tempo, etc.)
%         \item Influência de datasets na performance dos modelos
%         \item Métodos de ajuste de parâmetros e combinação de processamentos
%     \end{itemize}

%     \item \textbf{3. Metodologia}
%     \begin{itemize}
%         \item Definição das métricas para avaliar eficácia dos processamentos
%         \item Escolha do tipo de modelo de rede neural (classificação, detecção, regressão, etc.)
%         \item Seleção dos datasets para avaliação
%         \item Metodologia para combinação de processamentos unitários
%         \item Implementação de um método de ajuste automático de parâmetros
%     \end{itemize}

%     \item \textbf{4. Implementação dos Modelos}
%     \begin{itemize}
%         \item Construção de redes neurais para avaliação dos processamentos
%         \item Testes com diferentes arquiteturas e análise de variações nos resultados
%     \end{itemize}

%     \item \textbf{5. Coleta e Análise de Resultados}
%     \begin{itemize}
%         \item Impacto dos modelos no desempenho dos processamentos
%         \item Influência dos datasets nos resultados
%         \item Comparação entre diferentes estratégias de processamento
%     \end{itemize}

%     \item \textbf{6. Conclusão}
%     \begin{itemize}
%         \item Síntese dos resultados obtidos
%         \item Limitações e desafios encontrados
%         \item Sugestões para pesquisas futuras
%     \end{itemize}
% \end{itemize}

A seguir, é apresentado um cronograma de atividades para garantir a organização e a execução das tarefas.

\renewcommand{\arraystretch}{1.5}
\begin{table}[h!]
    \centering
    \resizebox{\textwidth}{!}{
    \begin{tabular}{|l|c|c|c|c|c|c|c|c|c|c|c|}
        \hline
        \textbf{Etapa} & \textbf{Fev} & \textbf{Mar} & \textbf{Abr} & \textbf{Mai} & \textbf{Jun} & \textbf{Jul} & \textbf{Ago} & \textbf{Set} & \textbf{Out} & \textbf{Nov} & \textbf{Dez} \\
        \hline
        \textbf{1. Introdução} & \checkmark &  &  &  &  &  &  &  &  &  &  \\
        \hline
        \textbf{2. Revisão Bibliográfica} & \checkmark & \checkmark & \checkmark & \checkmark & \checkmark &  &  &  &  &  &  \\
        \hline
        \textbf{3. Metodologia} &  & \checkmark & \checkmark & \checkmark & \checkmark & \checkmark & \checkmark &  &  &  &  \\
        \hline
        \textbf{4. Implementação dos Modelos} &  &  &  &  &  & \checkmark & \checkmark &  &  &  &  \\
        \hline
        \textbf{5. Coleta e Análise de Resultados} &  &  &  &  &  & \checkmark & \checkmark & \checkmark & \checkmark &  &  \\
        \hline
        \textbf{6. Conclusão e Redação Final} &  &  &  &  &  &  &  & \checkmark & \checkmark & \checkmark &  \\
        \hline
        \textbf{7. Defender} &  &  &  &  &  &  &  &  &  &  & \checkmark \\
        \hline
    \end{tabular}}
    \caption{Cronograma de Atividades}
    \label{tab:cronograma}
\end{table}