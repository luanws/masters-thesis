\documentclass[12pt]{article}
\usepackage[utf8]{inputenc}
\usepackage{amsmath}
\usepackage{graphicx}
\usepackage{geometry}

\geometry{a4paper, margin=1in}

\title{Metodologia para Comparação, Seleção, Combinação e Melhoramento de Processamentos de Imagem para Detecção e Classificação de Falhas em Isoladores}
\author{}
\date{}

\begin{document}

\maketitle

\section{Introdução}

A confiabilidade dos sistemas elétricos depende fortemente do estado de seus componentes, sendo os isoladores um dos elementos críticos para a segurança e estabilidade da rede. Defeitos em isoladores podem comprometer o funcionamento de equipamentos, gerar falhas no fornecimento de energia e aumentar os custos operacionais. Nesse contexto, técnicas de processamento de imagem surgem como ferramentas promissoras para a detecção e classificação de falhas nesses componentes. Entretanto, a diversidade de métodos disponíveis e a variação dos resultados obtidos tornam necessária uma abordagem sistemática para comparar, selecionar e aprimorar essas técnicas. Assim, este trabalho propõe o desenvolvimento de uma metodologia para a avaliação e melhoramento de processamentos de imagem voltados à detecção e classificação de falhas em isoladores elétricos.

\section{Proposta}

\subsection{Objetivo geral}

O objetivo geral deste estudo é desenvolver uma metodologia capaz de comparar, selecionar, combinar e aprimorar técnicas de processamento de imagem para a detecção e classificação de falhas em isoladores. Para alcançar esse objetivo, foram definidos os seguintes objetivos específicos:

\subsection{Objetivos específicos}

\begin{itemize}
    \item Estabelecer métricas para avaliar a eficácia dos processamentos de imagem, considerando aspectos como acurácia e tempo de processamento.
    \item Determinar o tipo de modelo de redes neurais ideal para avaliar o desempenho dos processamentos, podendo abranger classificação, detecção e regressão.
    \item Construir modelos de redes neurais destinados à avaliação do desempenho das técnicas de processamento de imagem, sem o intuito de encontrar um modelo definitivo.
    \item Analisar o impacto da escolha do modelo de rede neural no desempenho do processamento, considerando que diferentes modelos podem gerar diferentes resultados para um mesmo processamento.
    \item Avaliar a influência do dataset na eficácia do processamento, considerando possíveis variações nos resultados devido à utilização de diferentes conjuntos de dados.
    \item Desenvolver uma metodologia para o aprimoramento dos processamentos de imagem por meio da combinação de diferentes abordagens unitárias.
    \item Criar um método de ajuste automático de parâmetros dos processamentos de imagem, visando otimizar seus resultados sem a necessidade de intervenção manual extensa.
\end{itemize}

\subsection{Limitações}

A metodologia proposta será desenvolvida dentro de um conjunto de restrições previamente estabelecidas para garantir um escopo bem delimitado e viável dentro do período de realização da dissertação. São elas:

\begin{itemize}
    \item \textbf{Restrição a isoladores elétricos}: O estudo será focado exclusivamente na detecção e classificação de falhas em isoladores, não abrangendo outros componentes elétricos.
    \item \textbf{Uso de imagens previamente adquiridas}: Serão utilizadas imagens já disponíveis ou capturadas por métodos convencionais, sem o desenvolvimento de novas técnicas de aquisição de imagens.
    \item \textbf{Processamentos de imagem convencionais}: A metodologia será aplicada a técnicas de processamento de imagem já conhecidas, sem o desenvolvimento de novos algoritmos de base.
    \item \textbf{Modelos de redes neurais para avaliação}: O estudo se concentrará na avaliação do impacto dos modelos sobre os processamentos de imagem, sem o objetivo de encontrar um modelo definitivo para diagnóstico industrial.
    \item \textbf{Conjunto de datasets pré-existentes}: A análise será realizada utilizando conjuntos de dados já existentes ou obtidos por métodos convencionais, sem a criação de um novo dataset específico para o estudo.
    \item \textbf{Otimização limitada a ajustes de parâmetros}: A proposta contempla apenas a otimização dos parâmetros de técnicas existentes, sem o desenvolvimento de novas abordagens baseadas em inteligência artificial para otimização dos processamentos.
    \item \textbf{Avaliação em ambiente controlado}: O desempenho dos processamentos será analisado em condições simuladas, sem a realização de testes em ambientes industriais reais.
\end{itemize}

\end{document}
