% % % % % % % %  MDT UFSM 2021  % % % % % % % % 
%% Arquivo base para o documento - ver. 1.0 %%
% % % % % % % % % % % % % % % % % % % % % % % % 

% % % OPCOES DE COMPILACAO
% % % PAGINACAO
% % % PAGINACAO SIMPLES (FRENTE): PARA TRABALHOS COM MENOS DE 100 PAGINAS
\documentclass[oneside,openright,12pt]{ufsm_2021} %%%%% OPCAO PADRAO -> PAGINACAO SIMPLES. PARA TRABALHOS COM MAIS DE 100 PAGINAS COMENTE ESTA LINHA E DESCOMENTE A LINHA 
% % % % % % % % % % % % % % % % % % % % % % % % % % % % % % % % % % % % % % %
% PAGINACAO DUPLA (FRENTE E VERSO): PARA TRABALHOS COM MAIS DE 100 PAGINAS
% \documentclass[twoside,openright,12pt]{ufsm_2021}  %%%% PARA TRABALHOS COM MAIS DE 100 PAGINAS DESCOMENTE AQUI
% % % % % % % % % % % % % % % % % % % % % % % % % % % % % % % % % % % % %

% % % %  CODIFICACAO DO TEXTO 
% % % %  POR PADRAO USA-SE UTF8. PARA APLICAR A CODIFICACAO OESTE EUROPEU (ISO 8859-1) DESCOMENTE A LINHA ABAIXO. ELA ATIVA A OPCAO "latin1" DO PACOTE "inputenc"
% \oesteeuropeu
% % % % % % % % % % % 

% % % % % % % % PACOTES PESSOAIS % % % % % % % %  
\usepackage{lipsum}
\usepackage{quoting}

% % % % % % % % DEFINICOES PESSOAIS % % % % % % % %

% % % % % % % % % % % % % % % % % % % % % % % % % % % % % % % % % 
% % % % % % % % % % % % DADOS DO TRABALHO % % % % % % % % % % % % 
% % % % % % % % % % % % % % % % % % % % % % % % % % % % % % % % % 

% % % % % % % % % % INFORMACOES INSTITUCIONAIS % % % % % % % % % % 

% % CENTRO DE ENSINO DA UFSM
\centroensino{Centro de Ciências Naturais e Exatas}  %%% NOME POR EXTENSO
\centroensinosigla{CCNE}  %%% SIGLA

% % CURSO DA UFSM
\nivelensino{Pós-Graduação}  %%%%%%% NIVEL DE ENSINO 
\curso{Algum Curso}   %%%%% NOME POR EXTENSO
\ppg{PPGALGO}   %%%%%% SIGLA
\statuscurso{Programa}  %%%% STATUS= {Programa} ou {Curso}
% \EAD  %%%% para cursos EAD
% % % %  LOCAL DO CAMPUS OU POLO
\cidade{Santa Maria}
\estado{RS}


% % % % % % % % % % INFORMACOES DO AUTOR % % % % % % % % % % 
\author{Luan Willig Silveira}   %%%%% AUTOR DO TRABALHO
\sexo{M} %%%% SEXO DO AUTOR -> M=masculino   F=feminino (IMPORTANTE PARA AJUSTAR PAGINAS PRE-TEXTUAIS)
\grauensino{Mestrado}    %%%%%%%% GRAU DE ENSINO A SER CONCLUIDO
\grauobtido{Mestre}    %%%%% TITULO OBTIDO
\email{lalala@uhul.com}   %%%% E-MAIL PARA CATALOGRAFICA (COPYRIGHT) - OBRIGATORIO
% \endereco{Rua das abobrinhas, n. 666} %%%% TELEFONE PARA CATALOGRAFICA (COPYRIGHT) (CAMPO OPICIONAL -- CASO NAO POSSUA OU NAO QUEIRA DIVULGAR COMENTE A LINHA)
% \fone{11 2222 3333}   %%%% TELEFONE PARA CATALOGRAFICA (COPYRIGHT) FORMATO {11 2222 3333} (CAMPO OPICIONAL -- CASO NAO POSSUA OU NAO QUEIRA DIVULGAR COMENTE A LINHA)
% \fax{11 2222 3333}   %%%% FAX PARA CATALOGRAFICA (COPYRIGHT) FORMATO {11 2222 3333} (CAMPO OPICIONAL -- CASO NAO POSSUA OU NAO QUEIRA DIVULGAR COMENTE A LINHA)


% % % % % % % % % % INFORMACOES DA BANCA % % % % % % % % % % 
% OBSERVACOES: O CAMPO ORIENTADOR EH OBRIGATORIO E NAO DEVE SER COMENTADO
% % % % % %    OS DEMAIS MEMBROS DA BANCA (COOREIENTADOR E DEMAIS PROFESSORES) QUANDO COMENTADOS NAO APARECEM NA FOLHA DE APROVACAO (O LAYOUT DA FOLHA DE APROVACAO ESTA PREPARADO PARA O ORIENTADOR E ATE MAIS 4 MEMBROS NA BANCA
\orientador{João da Silva}{Dr}{AAAA}{M}{P}  %%%INFORMACOES SOBRE ORIENTADOR: OS CAMPOS SAO:{NOME}{SIGLA DA TITULACAO}{SIGLA DA INSTITUICAO DE ORIGEM}{SEXO} M=masculino   F=feminino {PARTE DA BANCA?} P=presidente  M=Membro  N=Nao faz parte
\coorientador{Maria da Costa}{Dra}{AAAA}{F}{M} %%%INFORMACOES SOBRE CO-ORIENTADOR: OS CAMPOS SAO:{NOME}{SIGLA DA TITULACAO}{SIGLA DA INSTITUICAO DE ORIGEM}{SEXO} M=masculino   F=feminino {PARTE DA BANCA?} P=presidente  M=Membro  N=Nao faz parte
\bancaum{Banca Um}{Dr}{AAAA}{F}{M}  %%%INFORMACOES SOBRE PRIMEIRO NOME DA BANCA: OS CAMPOS SAO:{NOME}{SIGLA DA TITULACAO}{SIGLA DA INSTITUICAO DE ORIGEM}{SEXO} M=masculino   F=feminino {PARTE DA BANCA?} P=presidente  M=Membro  N=Nao faz parte
%\bancadois{Banca Dois}{Dr}{BBBB}  %%%INFORMACOES SOBRE SEGUNDO NOME DA BANCA: OS CAMPOS SAO:{NOME}{SIGLA DA TITULACAO}{SIGLA DA INSTITUICAO DE ORIGEM}
%\bancatres{Banca Três}{Dra}{CCCC} %%%INFORMACOES SOBRE TERCEIRO NOME DA BANCA: OS CAMPOS SAO:{NOME}{SIGLA DA TITULACAO}{SIGLA DA INSTITUICAO DE ORIGEM}
%\bancaquatro{Banca Quatro}{Dr}{DDDD} %%%INFORMACOES SOBRE QUARTO NOME DA BANCA: OS CAMPOS SAO:{NOME}{SIGLA DA TITULACAO}{SIGLA DA INSTITUICAO DE ORIGEM}
%\bancacinco{Banca Cinco}{Dra}{EEEE} %%%INFORMACOES SOBRE QUARTO NOME DA BANCA: OS CAMPOS SAO:{NOME}{SIGLA DA TITULACAO}{SIGLA DA INSTITUICAO DE ORIGEM}
% \supervisor{Al Paccino}{Dr}{MAFIA}{M}{N} %%%INFORMACOES SOBRE SUPERVISOR (indicado para estagios): OS CAMPOS SAO:{NOME}{SIGLA DA TITULACAO}{SIGLA DA INSTITUICAO DE ORIGEM}{SEXO} M=masculino   F=feminino {PARTE DA BANCA?} P=Presidente  M=Membro  N=Nao faz parte



% % % % % % % % % % REALIZACAO POR VIDEO CONFERENCIA (MEMORANDO 04/2016 BIBLIOTECA CENTRAL UFSM)
\videoconferencia % % % % QUANDO O ACADEMICO DEFENDE POR VIDEO CONFERENCIA (PERMITIDO PELO ARTIGO 82 DO REGIMENTO GERAL DA PRPGP/UFSM). PARA DEFESAS NAS QUAIS O ACADEMICO ESTA PRESENTE COMENTE ESTA LINHA
% % % % QUANDO UM DOS MEMBROS DA BANCA PARTICIPA POR VIDEO CONFERENCIA INDICAR O MEMBRO DE ACORDO COM A LISTA ABAIXO. CASO CONTRARIO MANTER A PALAVRA "NAO". SAO PERMITIDOS, PELO REGIMENTO PRGPGP (ARTIGO 83) ATE 2 MEMBROS 
\videoconferenciabancap{NAO}  %%%% PRIMEIRO MEMBRO
\videoconferenciabancas{NAO}  %%%%% SEGUNDO MEMBRO
% % O > ORIENTADOR
% % CO > COORIENTADOR% % % % QUANDO UM DOS MEMBROS DA BANCA PARTICIPA POR VIDEO CONFERENCIA INDICAR O MEMBRO DE ACORDO COM A LISTA ABAIXO. CASO CONTRARIO MANTER A PALAVRA "NAO". SAO PERMITIDOS, PELO REGIMENTO PRGPGP ATE 2 MEMBROS.
% % 1 > BANCA UM
% % 2 > BANCA DOIS
% % 3 > BANCA TRÊS
% % 4 > BANCA QUATRO
% % 5 > BANCA CINCO
% % S > SUPERVISOR
% % % % % % % % % % % % % % % % % % % % % % % % % % % % % % % % % % % % 



% % % % % % % % % % INFORMACOES SOBRE O TRABALHO % % % % % % % % % %
% % % %  TITULO E SUBTITULO DO TRABALHO: ELES NÃO DEVEM ULTRAPASSAR, JUNTOS, 3 LINHAS NA COMPILAÇÃO DA CAPA. 
% SE O TRABALHO POSSUI SUBTITULO, ADICIONE ':' DENTRO DAS CHAVES ABAIXO 
\titulo{Título do trabalho em português:} %% NAO EH NECESSARIO CAPITALIZAR
% % % %  TITULO DO TRABALHO EM INGLES
% SE O TRABALHO POSSUI SUBTÍTULO, ADICIONE ':' DENTRO DAS CHAVES ABAIXO 
\englishtitle{Título do trabalho em inglês:}  %% NAO EH NECESSARIO CAPITALIZAR


% % % % O SUBTÍTULO É OPCIONAL, SE NÃO FOR USADO AS LINHAS ABAIXO DEVEM SER COMENTADAS

% SE O TRABALHO POSSUI SUBTÍTULO, ADICIONE ':' DENTRO DAS CHAVES ABAIXO 
\subtitulo{Subtítulo do trabalho em português} %% NAO EH NECESSARIO CAPITALIZAR
% % % %  SUB TITULO DO TRABALHO EM INGLES
\subenglishtitle{Subtítulo do trabalho em inglês}  %% NAO EH NECESSARIO CAPITALIZAR

% % % AREA DE CONCENTRACAO DO TRABALHO (CNPQ)
\areaconcentracao{Área de concentração do CNPq}
% % % TIPO DE TRABALHO - MANTER APENAS UMA LINHA DESCOMENTADA
\dissertacao  %% Tese de <nivel de ensino>
% \qualificacao %% Exame de Qualificação de <nivel de ensino>
% \dissertacao %% Dissertacao de <nivel de ensino>
% \monografia %% Monografia
% \monografiag  %% Monografia (nao exibe area de concentracao)
% \tf  %% Trabalho Final de <nivel de ensino>
% \tfg  %% Trabalho Final de Graduacao (nao exibe area de concentracao)
% \tcc  %% Trabalho de Conclusao de Curso
% \tccg  %% Trabalho de Conclusao de Curso (nao exibe area de concentracao)
% \relatorio  %% Relatório de Estágio (nao exibe area de concentracao)
% \generico   %%% Alternativa para aqueles cursos que nao recebem o titulo de bacharel ou licenciado. Ex: engenharia, arquitetura, etc... Os campos abaixo tambem devem ser preenchidos
%     \tipogenerico{Tipo de trabalho em português}
%     \tipogenericoen{Tipo de trabalho em inglês}
%     \concordagenerico{o}
%     \graugenerico{Engenheiro Eletricista}
% % % DATA DA DEFESA 
\data{15}{12}{2011} %% FORMATO {DD}{MM}{AAAA}


% % % % %  RESUMO E PALAVRAS CHAVE DO RESUMO - OBRIGATORIO PARA MDT-UFSM
\resumo{
Escreva seu resumo aqui! Você pode digitá-lo diretamente neste arquivo ou usar o comando input. O resumo deve ter apenas uma página, desde o cabeçalho até as palavras chave. Caso seu resumo seja maior, use comandos para diminuir espaçamento e fonte (até um mínimo de 10pt) no texto.  Segundo a MDT, é preciso que os resumos tenham, no máximo, 250 palavras para trabalhos de conclusão de curso de graduação, pós-graduação e iniciação científica e até 500 palavras para dissertações e teses.
}
\palavrachave{Palavra Chave 1. Palavra 2. Palavra 3. (...)}
% "... deverão constar, no mínimo, três palavras-chave, iniciadas em
% letras maiúsculas, cada termo separado dos demais por ponto, e
% finalizadas também por ponto." MDT 2012

% % % % %  ABSTRACT E PALAVRAS CHAVE DO RESUMO - OBRIGATORIO PARA MDT-UFSM
\abstract{
Write your abstract here! As recomendações do resumo também se aplicam ao abstract. \lipsum[0-1]
}
\keywords{Keyword 1. Keyword 2. Keyword 3. (...)}


% % %  ATIVACAO DE LISTAS E PAGINAS ESPECIAIS
% % %  PARA QUE APARECAO NAO NO TEXTO DESCOMENTE A LINHA ABAIXO -> POR PADRAO TODAS ESTAO ATIVIDADAS

% % LISTA DE FIGURAS 
\semfiguras   %%(QUANDO ATIVIDA NAO EXIBE A LISTA)
% % LISTA DE GRAFICOS 
\semgraficos   %%(QUANDO ATIVIDA NAO EXIBE A LISTA)
% % LISTA DE ILUSTRACOES 
\semilustracoes  %%(QUANDO ATIVIDA NAO EXIBE A LISTA)
% % LISTA DE TABELAS 
\semtabelas   %%(QUANDO ATIVIDA NAO EXIBE A LISTA)
% % LISTA DE QUADROS 
\semquadros   %%(QUANDO ATIVIDA NAO EXIBE A LISTA)
% % LISTA DE APENDICES 
\semapendices  %%(QUANDO ATIVIDA NAO EXIBE A LISTA)
% LISTA DE ANEXOS 
\semanexos   %%(QUANDO ATIVIDA NAO EXIBE A LISTA)


% % % FICHA CATALOGRAFICA
\semcatalografica  %%%%  (QUANDO ATIVIDA NAO EXIBE A FICHA CATALOGRAFICA NECESSITA DO ARQUIVO DA FICHA: ficha_catalografica.pdf
% % % A FICHA CATALOGRAFICA FORNECIDA PELA UFSM EH UM PDF DO TAMANHO A4
% % % EH POSSIVEL GERA-LA NO SITE http://cascavel.ufsm.br/ficha_catalografica/
% % % OS COMANDOS ABAIXO DEFINEM AS MARGENS PARA CORTAR A FICHA FORNECIDA E COLOCA-LA COMO UMA FIGURA NO DOCUMENTO LATEX
\margemesquerda{1.9}   %%%% CORTE DE MARGEM ESQUERDA EM CM
\margemdireita{1.5}   %%%% CORTE DE MARGEM DIREITA EM CM
\margemsuperior{2.75}  %%%% CORTE DE MARGEM SUPERIOR EM CM
\margeminferior{2.9} %%%% CORTE DE MARGEM INFERIOR EM CM
% % %  DICA: IMPRIMA UMA COPIA DA FICHA CATALOGRAFICA E FACA A MEDIDA DAS MARGENS!


% % % % % % % % % % % % % % % % % % % % % % % % % % % % % % % % % % % % % % 
% % % % % % % % % % % %  OPCOES DE FORMATACAO % % % % % % % % % % % % % % %
% % % % % % % % % % % % % % % % % % % % % % % % % % % % % % % % % % % % % % 

% % % % % % % % % % % % % % % % % % % % % % % % % % % % % % % % % % % % % %
% % % FONTES: descomente uma das opcoes. caso nenhuma seja ativada a clase usara a fonte padrao do latex

%% helvetica
\usepackage[scaled]{helvet}
\renewcommand*\familydefault{\sfdefault}

%% arial
% \renewcommand{\rmdefault}{phv} % Arial
% \renewcommand{\sfdefault}{phv} % Arial

%%times
% \usepackage{mathptmx}

\begin{document}

\pretextual

\chapter{Introdução}

\section{Proposta}

O presente estudo tem como objetivo desenvolver uma metodologia capaz de comparar, selecionar, combinar e aprimorar técnicas de processamento de imagem para a detecção e classificação de falhas em cadeias de isoladores. Para isso, serão estabelecidas métricas para avaliar a eficácia dos processamentos de imagem, considerando aspectos como acurácia e tempo de processamento. Além disso, serão construídos modelos de redes neurais para avaliar o desempenho dos processamentos, podendo abranger tarefas como classificação, detecção e regressão. No decorrer do estudo, serão construídos modelos de redes neurais voltados para a avaliação do desempenho das técnicas de processamento de imagem, sem a intenção de definir um modelo ideal.

Também será analisado o impacto da escolha do modelo de rede neural no desempenho do processamento, visto que diferentes modelos podem gerar resultados distintos para um mesmo processamento. A influência do dataset na eficácia do processamento será outro aspecto a ser investigado, considerando possíveis variações nos resultados devido ao uso de diferentes conjuntos de dados. Para aprimorar os processamentos de imagem, será desenvolvida uma metodologia que permita a combinação de diferentes processamentos unitários (processamentos de imagem que realizam uma única operação). Além disso, será criado um método de ajuste automático de parâmetros das técnicas de processamento de imagem, com o intuito de otimizar seus resultados sem exigir extensa intervenção manual.

A metodologia proposta será desenvolvida dentro de um conjunto de restrições previamente estabelecidas, garantindo um escopo bem delimitado e viável dentro do período de realização da dissertação. Primeiramente, o estudo será restrito à detecção e classificação de falhas em cadeias de isoladores elétricos, não abrangendo outros componentes elétricos. O uso de imagens previamente adquiridas será uma diretriz, de modo que apenas imagens já disponíveis ou capturadas por métodos convencionais serão utilizadas, sem o desenvolvimento de novas técnicas de aquisição de imagens. Além disso, a metodologia será aplicada exclusivamente a técnicas de processamento de imagem já conhecidas, sem a criação de novos algoritmos de base.

Os modelos de redes neurais desenvolvidos terão o propósito único de avaliar o impacto das redes sobre os processamentos de imagem, sem a intenção de definir um modelo definitivo para diagnóstico industrial. A análise será conduzida utilizando conjuntos de dados já existentes ou obtidos por métodos convencionais, sem a necessidade de criar um novo dataset específico para o estudo. A otimização contemplada estará limitada ao ajuste de parâmetros das técnicas existentes, não incluindo o desenvolvimento de novas abordagens baseadas em inteligência artificial para otimização dos processamentos. Por fim, toda a avaliação será realizada em ambiente controlado, sem a realização de testes em ambientes industriais reais.

A Figura \ref{fig:proposta} ilustra o diagrama da proposta de metodologia.

\begin{figure}[h]
    \label{fig:proposta}
    \centering
    \caption{Diagrama da proposta de metodologia}
    \includegraphics[width=\textwidth]{img/proposta.png}
    \fonte{Autor.}
\end{figure}

\section{Objetivo geral}

O objetivo geral deste estudo é desenvolver uma metodologia capaz de comparar, selecionar, combinar e aprimorar técnicas de processamento de imagem para a detecção e classificação de falhas em cadeias de isoladores.

\section{Objetivos específicos}

Para alcançar esse objetivo, foram definidos os seguintes objetivos específicos:

\begin{itemize}
    \item Estabelecer métricas para avaliar a eficácia dos processamentos de imagem, considerando aspectos como acurácia e tempo de processamento.
    \item Determinar o tipo de modelo de redes neurais ideal para avaliar o desempenho dos processamentos, podendo abranger classificação, detecção e regressão.
    \item Construir modelos de redes neurais destinados à avaliação do desempenho das técnicas de processamento de imagem, sem o intuito de encontrar um modelo definitivo.
    \item Analisar o impacto da escolha do modelo de rede neural no desempenho do processamento, considerando que diferentes modelos podem gerar diferentes resultados para um mesmo processamento.
    \item Avaliar a influência do dataset na eficácia do processamento, considerando possíveis variações nos resultados devido à utilização de diferentes conjuntos de dados.
    \item Desenvolver uma metodologia para o aprimoramento dos processamentos de imagem por meio da combinação de diferentes abordagens unitárias.
    \item Criar um método de ajuste automático de parâmetros dos processamentos de imagem, visando otimizar seus resultados sem a necessidade de intervenção manual extensa.
\end{itemize}

\section{Justificativa}

A crescente demanda por sistemas automatizados de inspeção de cadeias de isoladores evidencia a necessidade de técnicas avançadas de processamento de imagem para a detecção precoce de falhas. Conforme demonstrado por Gonzalez e Woods \cite{Gonzalez2008}, a análise digital de imagens permite extrair características relevantes para identificar anomalias em componentes elétricos, possibilitando diagnósticos mais precisos. Ademais, o emprego de redes neurais tem se destacado na resolução de problemas complexos de classificação e detecção, conforme ressaltado por LeCun et al. \cite{LeCun2015} e Krizhevsky et al. \cite{Krizhevsky2012}, contribuindo para a robustez dos sistemas de inspeção.

Estudos recentes apontam que a combinação de diferentes técnicas de processamento de imagem, aliada ao ajuste automático de parâmetros, pode resultar em melhorias significativas no desempenho dos sistemas de diagnóstico \cite{Li2019}. Assim, a proposta deste trabalho visa desenvolver uma metodologia que integre esses avanços, buscando não apenas aprimorar a acurácia e a eficiência dos processamentos, mas também possibilitar uma análise comparativa que leve em conta a influência de diferentes modelos e datasets.

Dessa forma, esta dissertação justifica-se pela necessidade de inovar na abordagem de detecção e classificação de falhas em cadeias de isoladores, promovendo ganhos práticos para a segurança e manutenção das redes elétricas, e contribuindo para a evolução do estado da arte em processamento de imagem e aprendizado de máquina.

\section{Cronograma}

% A seguir, apresenta-se a estrutura das etapas e subetapas que compõem o desenvolvimento deste trabalho. A lista a seguir organiza os principais tópicos a serem abordados ao longo da pesquisa, detalhando desde a introdução até a conclusão, incluindo as metodologias, a revisão bibliográfica, a implementação dos modelos e a análise dos resultados.

% \begin{itemize}
%     \item \textbf{1. Introdução}
%     \begin{itemize}
%         \item Justificativa com apresentação do problema e relevância
%         \item Objetivos gerais e específicos
%         \item Metodologia geral adotada
%         \item Estrutura da dissertação
%     \end{itemize}
    
%     \item \textbf{2. Revisão Bibliográfica}
%     \begin{itemize}
%         \item Processamento de imagens
%         \item Redes neurais para avaliação de processamentos de imagem
%         \item Métricas para análise de desempenho (acurácia, tempo, etc.)
%         \item Influência de datasets na performance dos modelos
%         \item Métodos de ajuste de parâmetros e combinação de processamentos
%     \end{itemize}

%     \item \textbf{3. Metodologia}
%     \begin{itemize}
%         \item Definição das métricas para avaliar eficácia dos processamentos
%         \item Escolha do tipo de modelo de rede neural (classificação, detecção, regressão, etc.)
%         \item Seleção dos datasets para avaliação
%         \item Metodologia para combinação de processamentos unitários
%         \item Implementação de um método de ajuste automático de parâmetros
%     \end{itemize}

%     \item \textbf{4. Implementação dos Modelos}
%     \begin{itemize}
%         \item Construção de redes neurais para avaliação dos processamentos
%         \item Testes com diferentes arquiteturas e análise de variações nos resultados
%     \end{itemize}

%     \item \textbf{5. Coleta e Análise de Resultados}
%     \begin{itemize}
%         \item Impacto dos modelos no desempenho dos processamentos
%         \item Influência dos datasets nos resultados
%         \item Comparação entre diferentes estratégias de processamento
%     \end{itemize}

%     \item \textbf{6. Conclusão}
%     \begin{itemize}
%         \item Síntese dos resultados obtidos
%         \item Limitações e desafios encontrados
%         \item Sugestões para pesquisas futuras
%     \end{itemize}
% \end{itemize}

A seguir, é apresentado um cronograma de atividades para garantir a organização e a execução das tarefas.

\renewcommand{\arraystretch}{1.5}
\begin{table}[h!]
    \centering
    \resizebox{\textwidth}{!}{
    \begin{tabular}{|l|c|c|c|c|c|c|c|c|c|c|c|}
        \hline
        \textbf{Etapa} & \textbf{Fev} & \textbf{Mar} & \textbf{Abr} & \textbf{Mai} & \textbf{Jun} & \textbf{Jul} & \textbf{Ago} & \textbf{Set} & \textbf{Out} & \textbf{Nov} & \textbf{Dez} \\
        \hline
        \textbf{1. Introdução} & \checkmark &  &  &  &  &  &  &  &  &  &  \\
        \hline
        \textbf{2. Revisão Bibliográfica} & \checkmark & \checkmark & \checkmark & \checkmark & \checkmark &  &  &  &  &  &  \\
        \hline
        \textbf{3. Metodologia} &  & \checkmark & \checkmark & \checkmark & \checkmark & \checkmark & \checkmark &  &  &  &  \\
        \hline
        \textbf{4. Implementação dos Modelos} &  &  &  &  &  & \checkmark & \checkmark &  &  &  &  \\
        \hline
        \textbf{5. Coleta e Análise de Resultados} &  &  &  &  &  & \checkmark & \checkmark & \checkmark & \checkmark &  &  \\
        \hline
        \textbf{6. Conclusão e Redação Final} &  &  &  &  &  &  &  & \checkmark & \checkmark & \checkmark &  \\
        \hline
        \textbf{7. Defender} &  &  &  &  &  &  &  &  &  &  & \checkmark \\
        \hline
    \end{tabular}}
    \caption{Cronograma de Atividades}
    \label{tab:cronograma}
\end{table}
\chapter{Trabalhos relacionados}

\section{Análise Sustentável da Detecção de Falhas em Isoladores Baseada em Otimização Visual Refinada}
O primeiro estudo aborda a detecção de falhas em isoladores em linhas de transmissão aéreas, destacando a vulnerabilidade desses componentes a fatores ambientais. A inspeção manual é ineficaz devido ao alto volume de dados e à complexidade dos fundos das imagens, levando à aplicação da rede neural convolucional de atenção regressiva (RA-CNN). O método proposto melhora a acurácia da detecção ao empregar extração de características em múltiplas escalas e operações recursivas, com otimização pelo algoritmo de Enxame de Partículas (PSO). Os resultados indicam que a RA-CNN (1+2+3) atinge 85,3\% de acurácia, superando os modelos FCAN e MG-CNN. Além disso, a abordagem proposta demonstra maior eficiência em tempo real, atingindo 25,4 FPS. \cite{wang2023}

\section{Detecção de Falhas em Isoladores em Imagens Aéreas de Linhas de Transmissão de Alta Voltagem Baseada em Modelo de Aprendizado Profundo}
O segundo estudo foca na detecção de falhas em isoladores por meio de imagens aéreas, utilizando um modelo YOLO modificado, denominado CSPD-YOLO, baseado no YOLO-v3 e na Rede Parcial de Estágio Cruzado. A pesquisa envolve a criação do conjunto de dados 'InSF-detection', composto por 1.331 imagens e 2.104 falhas rotuladas. O modelo CSPD-YOLO se destaca por uma alta acurácia (AP = 98,18\%) e eficiência no processamento (0,011 s), superando modelos tradicionais como YOLO-v3 e YOLO-v4. A análise qualitativa indica que o método é eficaz mesmo em cenários complexos, como presença de rios, vegetação e torres de energia. \cite{liu2021}

\section{Detecção de Defeitos em Isoladores por Imagem Baseada em Processamento Morfológico e Aprendizado Profundo}
O terceiro estudo propõe um método híbrido para detecção de defeitos em isoladores, combinando aprendizado profundo (Faster RCNN) com processamento morfológico. A segmentação das imagens utiliza técnicas de transformação de forma para identificação e separação de isoladores, enquanto a detecção de falhas é realizada por um modelo matemático aplicado a imagens binárias. O Faster RCNN alcança AP = 0,9175 e \textit{recall} = 0,98, superando abordagens baseadas em ResNet, YOLO e LBP+SVM. Além disso, a análise de desempenho em diferentes níveis de voltagem e condições de ruído demonstra a robustez do modelo \cite{zhang2022}.
O processo de segmentação realizado no trabalho é apresentado na Figura \ref{fig:segmentacao_zhang2022}.

\begin{figure}[H]
    \centering
    \caption{\label{fig:segmentacao_zhang2022}Segmentação de isoladores}
    \includegraphics[width=0.6\textwidth]{img/trabalhos_relacionados/segmentacao_zhang2022.png}
    \fonte{\citeonline{zhang2022}.}
\end{figure}

\section{Comparando redes neurais convolucionais e técnicas de pré- processamento para classificação de células HEp-2 em imagens de imunofluorescência}
A pesquisa avalia seis estratégias de pré-processamento e cinco arquiteturas de CNNs de última geração para classificar células HEp-2 em imagens de imunofluorescência, uma tarefa crítica em diagnósticos médicos. Métodos como aumento de dados (rotações, espelhamentos), ajuste fino e otimização de hiperparâmetros foram testados em conjunto com arquiteturas como Inception-V3 e ResNet. Surpreendentemente, o melhor desempenho, com 98,28\% de precisão, foi alcançado ao treinar o modelo Inception-V3 do zero, utilizando apenas aumento de dados sem pré-processamento adicional. A conclusão sugere que, para esse tipo de imagem, técnicas tradicionais de pré-processamento podem ser menos impactantes quando o aumento de dados é bem implementado, desafiando a necessidade de etapas complexas de preparação. A contribuição do estudo está em mostrar que, em cenários específicos como imagens médicas de imunofluorescência, estratégias simples podem superar abordagens mais elaboradas, oferecendo uma alternativa eficiente para aplicações práticas em classificação \cite{rodrigues2020comparing}.
A Figura \ref{fig:preprocessing_rodrigues2020comparing} apresenta os métodos e combinações de pré-processamentos utilizados no trabalho.

\begin{figure}[H]
    \centering
    \caption{\label{fig:preprocessing_rodrigues2020comparing}Etapas do método proposto}
    \includegraphics[width=0.6\textwidth]{img/trabalhos_relacionados/preprocessing_rodrigues2020comparing.png}
    \fonte{\citeonline{rodrigues2020comparing}.}
\end{figure}

\section{Efeitos do pré-processamento de imagens histopatológicas em redes neurais convolucionais}
O artigo analisa como diferentes níveis de pré-processamento afetam a classificação de imagens histopatológicas por CNNs, dividindo os dados em quatro categorias: imagens originais, pré-processadas normalmente (com redução de ruído e aprimoramento de células), outras pré-processadas normalmente e excessivamente pré-processadas (com operações morfológicas adicionais). Os experimentos revelam que o pré-processamento normal melhora a precisão ao remover ruídos de fundo e realçar características celulares, mas o excesso de processamento não agrega valor e pode até degradar o desempenho ao eliminar informações úteis. A conclusão enfatiza a importância de um equilíbrio no pré-processamento, recomendando ajustes moderados para maximizar a eficácia das CNNs em imagens histopatológicas. A contribuição do trabalho é fornecer uma análise comparativa detalhada que orienta pesquisadores e profissionais na escolha de técnicas de pré-processamento, evitando exageros que comprometam a qualidade dos dados em aplicações médicas \cite{ozturk2018histopathological}.

\section{O impacto de técnicas de pré-processamento e pós-processamento de imagens em frameworks de aprendizado profundo: uma revisão abrangente para análise de imagens de patologia digital}
Este trabalho explora como técnicas tradicionais de pré- e pós-processamento de imagens, como redução de ruído, correção de iluminação e segmentação, melhoram o desempenho de redes neurais em tarefas de patologia digital, incluindo classificação (tecido saudável vs. canceroso), detecção (contagem de linfócitos) e segmentação (núcleos e glândulas). Ao analisar uma ampla gama de estudos, os autores concluem que essas técnicas são indispensáveis para lidar com a variabilidade e complexidade das imagens médicas, melhorando significativamente a precisão e a robustez dos modelos de aprendizado profundo. A revisão destaca que o pós-processamento, como refinamento de contornos, também desempenha um papel crucial em tarefas de segmentação. A contribuição do artigo está em consolidar evidências sobre a eficácia dessas abordagens, oferecendo um guia abrangente para pesquisadores que buscam integrar métodos tradicionais ao treinamento de redes neurais, especialmente em contextos de patologia digital onde a qualidade dos dados é crítica \cite{Salvi2021}.

\section{Resumo dos Trabalhos}

Diversos estudos abordam o impacto do pré-processamento de imagens na análise por redes neurais convolucionais (CNNs) e outros modelos de aprendizado profundo. O estudo de \citeonline{liu2021} e \citeonline{wang2023} optam por não realizar processamentos significativos, utilizando apenas redimensionamento e normalização.

Por outro lado, \citeonline{ozturk2018histopathological} investigam diferentes algoritmos de pré-processamento, incluindo remoção de fundo, filtros de suavização e equalização de histograma, além de um método de sobre-processamento baseado em limiar adaptativo. \citeonline{rodrigues2020comparing} testam técnicas como redimensionamento, alongamento de contraste, equalização de histograma e subtração da média, constatando que o uso de imagens originais favorece o desempenho da CNN, enquanto o data augmentation tem impacto positivo. \citeonline{Salvi2021} destacam que técnicas como remoção de artefatos, normalização de cor e seleção de patches melhoram a precisão dos modelos e reduzem o tempo computacional. Por fim, \citeonline{zhang2022} exploram a segmentação de isoladores para otimizar a classificação.

\section{Justificativa da Relevância da Metodologia Proposta}

Os trabalhos analisados demonstram a importância do pré-processamento na análise de imagens, mas também indicam que determinadas abordagens podem comprometer o desempenho da CNN. Em especial, \citeonline{rodrigues2020comparing} evidenciam que a eliminação de ruídos e artefatos pode não ser sempre benéfica. Além disso, \citeonline{Salvi2021} reforçam que técnicas de segmentação e normalização podem aprimorar a análise quando aplicadas corretamente. No entanto, nenhum dos estudos analisados aborda a metodologia específica proposta nesta dissertação, o que destaca sua inovação e potencial contribuição para a área.

\section{Tabela Comparativa dos Trabalhos}

A tabela \ref{tab:comparacao_de_trabalhos_relacionados} compara os resultados de diferentes estudos sobre pré-processamento de imagens e seu impacto nos modelos de classificação de imagens. Os estudos variam desde melhorias no desempenho até riscos de sobre-processamento, destacando a falta de consenso e a necessidade de novas abordagens. A metodologia proposta neste trabalho busca preencher essa lacuna ao introduzir um método inovador para determinar os processamentos mais eficientes e otimizar os parâmetros de processamento, oferecendo uma solução mais robusta e adaptável para análise de imagens.

\begin{table}[h]
\centering
\resizebox{\textwidth}{!}{
\begin{tabular}{|l|p{11cm}|}
\hline
\textbf{Trabalho} & \textbf{Resultado do pré-processamento} \\
\hline
\citeonline{liu2021} & Sem impacto significativo \\
\hline
\citeonline{ozturk2018histopathological} & Melhorou contraste, mas risco de sobre-processamento \\
\hline
\citeonline{rodrigues2020comparing} & Afetou negativamente a CNN; data augmentation foi positivo \\
\hline
\citeonline{Salvi2021} & Melhorou precisão e reduziu tempo computacional \\
\hline
\citeonline{wang2023} & Sem impacto significativo \\
\hline
\citeonline{zhang2022} & Melhorou o desempenho do modelo \\
\hline
\end{tabular}
}
\caption{Comparação dos trabalhos relacionados}
\label{tab:comparacao_de_trabalhos_relacionados}
\end{table}

A análise desses estudos reforça a lacuna existente na literatura e a necessidade de uma nova abordagem, como a metodologia proposta nesta dissertação.
A proposta de um método para determinar os processamentos mais eficientes e otimizar os parâmetros de processamento representa uma contribuição significativa para a área de análise de imagens. Através da combinação de técnicas de pré-processamento adaptativo e otimização de parâmetros, o trabalho busca não apenas melhorar o desempenho dos modelos de aprendizado profundo, mas também oferecer uma solução prática e eficiente para cenários complexos. Essa abordagem pode impactar positivamente diversas áreas, como diagnóstico médico, inspeção industrial e análise de imagens aéreas, ao proporcionar resultados mais precisos e confiáveis. Além disso, a metodologia proposta pode servir como base para futuras pesquisas e aplicações em diferentes contextos, ampliando as possibilidades de utilização das redes neurais em tarefas desafiadoras de classificação e detecção de objetos em imagens.

\section{Considerações finais do capítulo}

A metodologia apresentada neste trabalho se diferencia dos estudos revisados ao introduzir um novo enfoque que não foi explorado anteriormente. Enquanto os trabalhos existentes se concentram em construção de modelos de redes neurais e utilização de técnicas tradicionais de pré-processamento e normalização, a metodologia deste trabalho propõe a criação de um método para determinar os processamentos mais eficientes das imagens, além de um método de otimização dos parâmetros de processamento. Além disso, a pesquisa busca integrar novas abordagens que possam aprimorar a análise de dados em contextos variados, contribuindo para a evolução das técnicas de aprendizado profundo. A implementação dessas novas abordagens poderá oferecer insights valiosos para futuras investigações e aplicações práticas.
\chapter{Revisão Bibliográfica}

O processamento de imagens é uma ferramenta essencial para garantir a confiabilidade do Sistema Elétrico de Potência (SEP), especialmente na detecção e classificação de falhas em equipamentos de linhas de transmissão de energia elétrica. Essa técnica permite identificar problemas em componentes como isoladores, fixadores e suportes, que, se não tratados, podem causar interrupções no fornecimento de energia. O uso de imagens capturadas por drones ou câmeras especiais facilita a inspeção de grandes extensões de linhas de transmissão, reduzindo custos e aumentando a segurança ao evitar a necessidade de intervenções manuais em locais de difícil acesso \cite{eze2022deep}.

Métodos avançados de análise de imagens, como os baseados em aprendizado profundo, ajudam a reconhecer padrões que indicam falhas, mesmo em condições adversas, como baixa visibilidade ou equipamentos desgastados \cite{Altaie2023}. Essas abordagens são particularmente úteis em regiões com infraestrutura antiga, onde a manutenção regular é desafiadora. Além disso, o processamento de imagens possibilita uma resposta rápida a problemas, minimizando o impacto de falhas na rede elétrica e melhorando a continuidade do serviço \cite{kumar2023novel}.

A automação proporcionada pelo processamento de imagens também contribui para a eficiência operacional. Técnicas modernas permitem monitorar equipamentos em tempo real, identificando danos antes que se tornem críticos \cite{eze2022deep}. Isso é crucial para manter a estabilidade do SEP, especialmente em áreas remotas ou com alta demanda energética. Assim, o processamento de imagens não apenas aprimora a manutenção das linhas de transmissão, mas também reforça a segurança e a confiabilidade do fornecimento de energia elétrica.

A seguir, será apresentada uma revisão dos principais conceitos e técnicas de processamento de imagens, que podem ser aplicados na detecção e classificação de falhas em equipamentos de linhas de transmissão de energia elétrica.

\section{Processamento de Imagens}

O processamento de imagens desempenha um papel fundamental no contexto do Sistema Elétrico de Potência (SEP), especialmente em atividades de inspeção, manutenção preditiva e monitoramento de ativos em linhas de transmissão. Com o uso crescente de drones, câmeras térmicas e sensores ópticos, a obtenção de imagens de componentes da rede elétrica tornou-se mais acessível e eficiente. No entanto, a qualidade e a variabilidade dessas imagens exigem técnicas robustas de pré-processamento para garantir resultados precisos em tarefas como a detecção de falhas, corrosões, aquecimentos anômalos e objetos estranhos nas estruturas. Esta seção apresenta os principais métodos de processamento de imagens empregados para preparar dados visuais que serão utilizados em modelos baseados em aprendizado de máquina e redes neurais, contribuindo diretamente para a confiabilidade, segurança e eficiência operacional do SEP.

\subsection{Normalização}
A normalização é uma etapa fundamental no pré-processamento de imagens para redes neurais, pois padroniza os valores dos pixels, facilitando a convergência durante o treinamento e melhorando a generalização do modelo. Um método comum é a normalização de valores de pixels, que escala os valores para intervalos como [0,1] ou [-1,1], frequentemente realizada dividindo os valores originais pelo máximo possível (por exemplo, 255 para imagens de 8 bits) \cite{sharma2024deep}. Outro método é a normalização Z-score, que subtrai a média dos pixels e divide pelo desvio padrão, resultando em dados com média zero e variância unitária \cite{chen2023robustness}. A equalização de histograma também é utilizada para redistribuir as intensidades dos pixels, aumentando o contraste e destacando detalhes em imagens de baixa qualidade \cite{chen2023robustness}. Além disso, a padronização de cores, como subtrair os valores médios dos canais RGB, centraliza os dados em torno de uma distribuição normal, o que é particularmente útil para redes convolucionais \cite{sciencedirect2023normalization}. Técnicas mais avançadas, como a normalização por percentis, utilizam o 5º e o 95º percentis como limites para lidar com valores discrepantes, enquanto a correspondência de histogramas ajusta a distribuição de intensidades com base em pontos de referência \cite{isola2023comparison}. Essas abordagens garantem que as redes neurais processem dados de forma consistente, reduzindo a sensibilidade a variações de iluminação ou escala, especialmente em tarefas de visão computacional \cite{sharma2024deep}.

A Figura \ref{fig:normalizacao} ilustra o processo de normalização de imagens, onde a imagem original é transformada em uma imagem normalizada, facilitando a extração de características relevantes.

\begin{figure}[H]
    \centering
    \caption{\label{fig:normalizacao}Normalização de Imagens}
    \includegraphics[width=0.8\textwidth]{img/revisao_bibliografica/normalizacao.png}
    \fonte{\citeonline{kuehlkamp2013ferramenta}.}
\end{figure}

\subsection{Redimensionamento e Recorte}
O redimensionamento é essencial para ajustar as imagens ao tamanho de entrada esperado pelas arquiteturas de redes neurais, garantindo compatibilidade e consistência. Um método comum é redimensionar as imagens para um tamanho fixo, como 224x224 pixels, amplamente utilizado em modelos como ResNet e VGG \cite{chen2023robustness}. Isso pode ser feito por meio de interpolação bilinear ou bicúbica, que suaviza as transições entre pixels, embora métodos mais avançados, como interpolação baseada em Fourier, também sejam explorados \cite{dennanni2019resizing}. O recorte, por outro lado, extrai uma região de interesse da imagem, frequentemente centrada, para preservar áreas relevantes, especialmente quando as dimensões originais variam significativamente \cite{sciencedirect2023normalization}. Estudos indicam que o redimensionamento para tamanhos menores pode acelerar o treinamento, mas tamanhos muito reduzidos podem comprometer a qualidade das características extraídas \cite{sabottke2020effect}. Além disso, o recorte aleatório é usado em conjunto com aumento de dados para introduzir variabilidade durante o treinamento \cite{nalepa2022data}. Essas técnicas são cruciais para lidar com conjuntos de dados heterogêneos, garantindo que as entradas sejam uniformes sem perda significativa de informação \cite{chen2023robustness}.

A Figura \ref{fig:redimensionamento_e_recorte} ilustra o processo de redimensionamento e recorte de imagens, na qual a imagem da esquerda (original) é utilizada para extrair uma região de interesse (recorte) e em seguida redimensionada para um tamanho fixo (imagem da direita).

\begin{figure}[H]
    \centering
    \caption{\label{fig:redimensionamento_e_recorte}Redimensionamento e recorte}
    \includegraphics[width=1\textwidth]{img/revisao_bibliografica/redimensionamento_e_recorte.png}
    \fonte{Adaptado de \citeonline{venturelli2021}.}
\end{figure}

\subsection{Aumento de Dados}
O aumento de dados é uma estratégia poderosa para ampliar a diversidade do conjunto de treinamento, reduzindo o risco de sobreajuste e melhorando a robustez do modelo. Técnicas geométricas incluem espelhamento horizontal ou vertical, rotações em ângulos variados (de 1° a 359°), translações, cortes aleatórios e ajustes de escala, que simulam diferentes perspectivas e tamanhos \cite{shorten2019survey}. Transformações no espaço de cores, como ajustes de brilho, contraste, saturação e matiz, ajudam a lidar com variações de iluminação \cite{shorten2019survey}. Métodos mais avançados, como apagamento aleatório, mascaram partes da imagem para simular oclusões, enquanto a mistura de imagens combina pixels de diferentes amostras para criar novas instâncias \cite{shorten2019survey}. Por exemplo, o método SamplePairing reduziu o erro no conjunto CIFAR-10 de 8,22\% para 6,93\% \cite{shorten2019survey}. Além disso, redes adversárias generativas (GANs) são usadas para gerar imagens sintéticas, especialmente em domínios com dados limitados, como imagens médicas, alcançando melhorias de até 10\% em precisão \cite{shorten2019survey}. Essas técnicas são particularmente valiosas em cenários com poucos dados, permitindo que as redes neurais generalizem melhor para condições não vistas \cite{nalepa2022data}.

\subsection{Redução de Ruído}
A redução de ruído remove interferências que podem comprometer o desempenho das redes neurais, sendo especialmente crítica em aplicações como imagens médicas e vigilância. Métodos tradicionais, como filtros de média ou mediana, são complementados por abordagens baseadas em aprendizado profundo, como redes neurais convolucionais (CNNs) especializadas, como DnCNNs, que aprendem a mapear imagens ruidosas para versões limpas \cite{sharma2024deep}. Autoencoders também são empregados para reconstruir imagens a partir de representações latentes, eliminando ruídos como Gaussianos ou de sal e pimenta \cite{sharma2024deep}. Técnicas como Total Variation Denoising (TVD) e Non-Local Means (NLM) exploram regularizações e similaridades entre pixels para preservar detalhes \cite{sharma2024deep}. Um estudo demonstrou que a aplicação de DnCNNs em imagens de tomografia computadorizada resultou em uma precisão de detecção de câncer de pulmão variando de 86,17\% a 99,67\% \cite{sharma2024deep}. Além disso, métodos baseados em redes neurais profundas, como o Deep Neural Filter (DNF), alcançaram melhorias de até 10 dB na relação sinal-ruído em sinais de EEG \cite{peer2022real}. Essas abordagens são essenciais para garantir que as redes neurais processem imagens de alta qualidade, minimizando artefatos que poderiam obscurecer características críticas \cite{sharma2024deep}.

A Figura \ref{fig:reducao_de_ruido} ilustra o processo de redução de ruído, onde a imagem original (à esquerda) é processada para remover o ruído, resultando em uma imagem mais limpa (à direita).

\begin{figure}[H]
    \centering
    \caption{\label{fig:reducao_de_ruido}Redução de Ruído}
    \includegraphics[width=1\textwidth]{img/revisao_bibliografica/reducao_de_ruido.png}
    \fonte{Adaptado de \citeonline{wavelet_denoising}.}
\end{figure}

\subsection{Ajuste de Contraste e Brilho}
O ajuste de contraste e brilho melhora a visibilidade das características das imagens, sendo crucial para tarefas que dependem de detalhes finos. A equalização de histograma redistribui as intensidades dos pixels para maximizar o contraste, enquanto a equalização adaptativa limitada por contraste (CLAHE) evita a amplificação excessiva de ruído em regiões homogêneas \cite{sciencedirect2023normalization}. A correção gama ajusta a curva de intensidade para realçar detalhes em áreas escuras ou claras, sendo amplamente usada em imagens de baixa qualidade \cite{sciencedirect2023normalization}. Métodos baseados em aprendizado profundo, como redes convolucionais fuzzy, integraram filtros Gaussianos e triangulares para melhorar imagens de íris, alcançando até 97\% de precisão em tarefas de reconhecimento \cite{sharma2024deep}. Além disso, técnicas como RetinexDIP foram propostas para melhorar a resolução e reduzir o consumo de memória em comparação com métodos tradicionais \cite{sharma2024deep}. Essas abordagens são fundamentais para preparar imagens para redes neurais, garantindo que as características relevantes sejam destacadas \cite{sciencedirect2023normalization}.

% imagem de exemplo do CLAHE. cite isso no texto abaixo
A Figura \ref{fig:clahe} ilustra o efeito do CLAHE em uma imagem, destacando detalhes que antes estavam obscurecidos.

\begin{figure}[H]
    \centering
    \caption{\label{fig:clahe}Ajuste de Contraste e Brilho com CLAHE}
    \includegraphics[width=1\textwidth]{img/revisao_bibliografica/clahe.png}
    \fonte{\citeonline{pandey2023image}.}
\end{figure}

\subsection{Aumento de Nitidez}
O aumento de nitidez realça bordas e detalhes finos, facilitando tarefas como detecção de objetos e segmentação. Técnicas tradicionais, como a máscara de desfoque, aplicam filtros de alta passagem para enfatizar transições de intensidade \cite{sciencedirect2023normalization}. Métodos baseados em redes neurais, como CNNs, foram desenvolvidos para detectar e corrigir nitidez, como no caso da detecção de máscaras de desfoque (USM), superando métodos baseados em codificação ternária perpendicular a bordas \cite{ding2018detecting}. Em aplicações específicas, como imagens de documentos, redes convolucionais combinadas com filtros de Gabor e desfoque melhoraram a legibilidade, reduzindo distorções como sombras e ruídos \cite{ben2022deep}. Essas técnicas são particularmente úteis em cenários onde a clareza das bordas é essencial para o desempenho do modelo \cite{sharma2024deep}.

A Figura \ref{fig:aumento_de_nitidez} ilustra o efeito do aumento de nitidez em uma imagem, onde os detalhes são mais evidentes após o processamento.

\begin{figure}[H]
    \centering
    \caption{\label{fig:aumento_de_nitidez}Aumento de Nitidez}
    \includegraphics[width=1\textwidth]{img/revisao_bibliografica/aumento_de_nitidez.png}
    \fonte{Adaptado de \citeonline{joshi2025}.}
\end{figure}

\subsection{Conversão de Espaço de Cores}
A conversão de espaço de cores adapta as imagens às necessidades específicas da tarefa, simplificando o processamento ou destacando características relevantes. A conversão de RGB para escala de cinza reduz a dimensionalidade, sendo útil em tarefas onde a cor não é essencial \cite{sharma2024deep}. Espaços como HSV e LAB são preferidos em aplicações que requerem separação de matiz, saturação ou luminância, como segmentação de objetos \cite{sharma2024deep}. Redes neurais também foram usadas para realizar conversões de espaço de cores, como de RGB para XYZ, alcançando erros de cor inferiores a 1,0 unidade ΔE 2000 em mais de 85\% dos casos testados \cite{macdonald2019color}. Essas conversões são valiosas para otimizar a extração de características e reduzir a complexidade computacional em tarefas de visão computacional \cite{sharma2024deep}.

\subsection{Restauração e Desembaçamento de Imagens}
A restauração de imagens visa recuperar a imagem original a partir de versões degradadas por desfoque, ruído ou outras distorções. O desembaçamento, um subcampo da restauração, utiliza redes neurais como U-Net para corrigir desfoques dinâmicos, alcançando PSNR de 31,53 no conjunto GoPro e 31,32 no Real Blur \cite{Lian2023Deblurring}. Métodos baseados em autoencoders convolucionais foram propostos para restaurar imagens em aplicações de fotografia computacional e sensoriamento remoto \cite{barreto2020cnn}. Além disso, redes neurais como DnCNNs foram aplicadas para remover ruídos específicos, como speckle em imagens holográficas \cite{sharma2024deep}. Essas técnicas são cruciais para preparar imagens de alta qualidade para redes neurais, especialmente em domínios onde a clareza é essencial \cite{sumida2019deep}.

A Figura \ref{fig:desembacamento} ilustra o processo de desembaçamento, onde a imagem original (à esquerda) é processada para remover o desfoque, resultando em uma imagem mais nítida (à direita).

\begin{figure}[H]
    \centering
    \caption{\label{fig:desembacamento}Restauração e Desembaçamento de Imagens}
    \includegraphics[width=1\textwidth]{img/revisao_bibliografica/desembacamento.png}
    \fonte{Adaptado de \citeonline{mathworks2025lucyrichardson}.}
\end{figure}

\subsection{Detecção de Bordas}
A detecção de bordas identifica limites e formas nas imagens, sendo uma etapa fundamental em muitas tarefas de visão computacional. Redes neurais, como redes de codificação-decodificação, foram desenvolvidas para detectar bordas com alta precisão, superando detectores tradicionais como Canny em imagens ruidosas \cite{yu1994edge}. Métodos inspirados em mecanismos biológicos, como redes com atenção seletiva, melhoraram a extração de características globais, resultando em mapas de bordas mais robustos \cite{chen2022edge}. Essas abordagens são essenciais para pré-processar imagens, fornecendo informações estruturais que facilitam a segmentação e o reconhecimento de objetos \cite{yu1994edge}.

A Figura \ref{fig:deteccao_de_bordas} ilustra o processo de detecção de bordas, onde as bordas da imagem original (à esquerda) são destacadas na imagem processada (à direita).

\begin{figure}[H]
    \centering
    \caption{\label{fig:deteccao_de_bordas}Detecção de Bordas}
    \includegraphics[width=1\textwidth]{img/revisao_bibliografica/deteccao_de_bordas.png}
    \fonte{\citeonline{couto2024regions}.}
\end{figure}

\subsection{Correção de Iluminação}
A correção de iluminação normaliza as condições de luz nas imagens, garantindo consistência na extração de características. Métodos baseados em aprendizado profundo, como redes convolucionais, foram propostos para corrigir imagens com iluminação desigual, como pinturas, alcançando resultados superiores em métricas como NIQE e LOE \cite{li2020simple}. Técnicas híbridas que combinam modelos baseados em aprendizado e físicos foram aplicadas para melhorar a detecção de objetos em condições de luz variada, como em imagens de plantações \cite{yang2022using}. Essas abordagens são particularmente úteis em cenários onde a iluminação não uniforme pode comprometer o desempenho do modelo \cite{li2020simple}.

A Figura \ref{fig:correcao_de_iluminacao} ilustra o processo de correção de iluminação, onde a imagem original (à esquerda) é processada para uniformizar a iluminação, resultando em uma imagem mais equilibrada (à direita).

\begin{figure}[H]
    \centering
    \caption{\label{fig:correcao_de_iluminacao}Correção de Iluminação}
    \includegraphics[width=1\textwidth]{img/revisao_bibliografica/correcao_de_iluminacao.png}
    \fonte{\citeonline{Bascle2006IlluminationCorrection}.}
\end{figure}

\subsection{Super-Resolução}
A super-resolução aumenta a resolução de imagens, gerando versões de alta qualidade a partir de entradas de baixa resolução. Redes neurais, como redes convolucionais profundas e redes adversárias generativas (GANs), alcançaram resultados impressionantes, com modelos como SRGAN produzindo imagens fotorrealistas \cite{ledig2017photo}. Em aplicações biológicas, redes como DPA-TISR foram desenvolvidas para imagens de células vivas, alcançando fidelidade temporal e consistência em mais de 10.000 pontos temporais \cite{liu2025neural}. Essas técnicas são valiosas para tarefas que requerem detalhes finos, como análise médica e vigilância, permitindo que redes neurais processem imagens com maior clareza \cite{ledig2017photo}.

\subsection{Conclusão parcial da seção}

O processamento de imagens é uma etapa crucial para garantir a eficácia dos modelos de aprendizado profundo aplicados ao Sistema Elétrico de Potência (SEP). As técnicas discutidas, como normalização, redimensionamento, aumento de dados e redução de ruído, podem ser fundamentais para preparar as imagens antes de serem alimentadas em redes neurais. Essas abordagens não apenas melhoram a qualidade das imagens, mas também garantem que os modelos sejam mais robustos e capazes de generalizar em diferentes condições. A escolha adequada dessas técnicas pode impactar significativamente o desempenho dos modelos na detecção e classificação de falhas em equipamentos de linhas de transmissão.

\section{Redes Neurais}
Uma rede neural artificial é formada por um grande número de neurônios para funcionar corretamente, mas para compreender o funcionamento de uma rede neural, deve-se definir o modelo de um único neurônio artificial. Esse modelo é apresentado na Figura~\ref{fig:neuronio}.

\begin{figure}[H]
    \centering
    \caption{\label{fig:neuronio}Modelo de um Neurônio Artificial}
    \includegraphics[width=0.8\textwidth]{img/revisao_bibliografica/neuronio.png}
    \fonte{\citeonline{braga2011redes}.}
\end{figure}

Os passos para a obtenção da saída de um neurônio artificial são:

\begin{enumerate}
    \item O modelo recebe um número $m$ de entradas $x_1, x_2, ..., x_m$;
    \item Cada uma dessas entradas é multiplicada por um peso: $x_1w_1, x_2w_2, ..., x_mw_m$;
    \item Somam-se as entradas multiplicadas pelos seus respectivos pesos: $\sum_{n=1}^{m} w_n x_n$;
    \item Adiciona-se o bias: $b + \sum_{n=1}^{m} w_n x_n$;
    \item O resultado passa por uma função de ativação: $\varphi \left( b + \sum_{n=1}^{m} w_n x_n \right)$.
\end{enumerate}

Seguidos os passos, a equação de saída de um neurônio artificial é descrita pela Equação~\ref{eq:neuronio}.

\begin{equation}
    y = \varphi \left( b + \sum_{n=1}^{m} w_n x_n \right)
    \label{eq:neuronio}
\end{equation}

Visto o modelo de um único neurônio artificial, o conceito de rede neural composta pela associação de diversos neurônios é descrito a seguir.

\subsection{Tipos de redes neurais}

Existem diversos tipos de redes neurais que se distinguem tanto em termos de seus princípios de funcionamento quanto em suas aplicações práticas específicas. A seguir, serão discutidos alguns desses tipos de redes neurais de maneira individualizada, a fim de fornecer uma compreensão mais aprofundada sobre seu funcionamento e suas aplicações \cite{alex2020}.

\subsubsection{Perceptron (P), Feed Forward Network (FFN)}

FFNs são o tipo mais básico de rede neural, em que a informação flui linearmente até a saída e cada neurônio realiza uma operação matemática linear do tipo $wx + b$, sendo $x$ o valor de entrada, $w$ o peso e $b$ o bias do neurônio. O resultado passa por uma função de ativação e em seguida é enviado para a próxima camada. As redes neurais do tipo FFN (exemplo mostrado através da Figura \ref{fig:ffn}) possuem conexões em apenas um único sentido, geralmente limitadas a 5 camadas \cite{alex2020}.

\begin{figure}[H]
    \centering
    \caption{\label{fig:ffn}FFN}
    \includegraphics[width=0.8\textwidth]{img/revisao_bibliografica/ffn.png}
    \fonte{\citeonline{alex2020}.}
\end{figure}

As redes do tipo P, Figura \ref{fig:perceptron}, são um caso especial de uma rede FFN, em que todos os neurônios de uma camada são conectados com todos neurônios da camada seguinte \cite{alex2020}.

\begin{figure}[H]
    \centering
    \caption{\label{fig:perceptron}Perceptron}
    \includegraphics[width=0.8\textwidth]{img/revisao_bibliografica/perceptron.png}
    \fonte{\citeonline{alex2020}.}
\end{figure}

As FFNs são utilizadas para problemas em que os dados de entrada têm impacto atemporal nos dados de saída, em que a saída não depende do estado anterior da rede neural. Um exemplo é usar informações de um exame de sangue para determinar a presença de uma doença.

\subsubsection{Convolutional neural network (CNN) ou Deep convolutional network (DCN)}

As redes neurais estudadas até então, não consideram uma relação de vizinhança entre os dados de entrada. Por exemplo, não faria a menor diferença se antes do treinamento a posição de todos os dados de entrada fossem embaralhadas da mesma forma. Porém, essa relação de vizinhança pode ser muito importante para alguns casos específicos como no reconhecimento de imagens, reconhecimento de voz, análise grafista do mercado financeiro, etc. No reconhecimento de imagens, por exemplo, grande parte da informação está contida na relação de vizinhança dos pixels como o contraste e a textura.

Uma CNN, cuja estrutura está representada através da Figura \ref{fig:cnn}, percebe uma imagem como uma caixa retangular cuja largura e altura são medidas pelo número de pixels da imagem e a profundidade é representada por cada uma das três camadas de cores, vermelho, verde e azul referidas como canais \cite{veen2016}.

\begin{figure}[H]
    \centering
    \caption{\label{fig:cnn}CNN}
    \includegraphics[width=0.8\textwidth]{img/revisao_bibliografica/cnn.png}
    \fonte{\citeonline{veen2016}.}
\end{figure}

Ao longo das camadas de uma rede CNN, as dimensões da imagem se alteram, pois a medida em que a altura e largura da imagem diminuem, o número de canais aumenta, reduzindo o volume de dados. Esse processo é chamado de pooling, que faz um resumo dos dados através do descarte das saídas menos significativas, mantendo somente às de maior valor \cite{veen2016}.

O processo de convolução é realizado arrastando uma janela (kernel) de dimensão menor pela imagem original, sendo essa janela uma rede FFN \cite{veen2016}. Por exemplo, se uma imagem 5x5 pixels passar pelo processo da convolução e supondo uma janela de 3x3 com passo 1 (stride), primeiramente os 3x3 pixels do canto superior esquerdo da imagem original passarão por uma FFN. Em seguida, essa janela é arrastada 1 pixel (tamanho do passo) para a direita e o processo se repete até ao final da imagem.

\subsection{Funções de ativação}

As funções de ativação introduzem não-linearidade nas redes neurais, permitindo que elas aprendam relações complexas entre entradas e saídas. Sem essas funções, mesmo com múltiplas camadas, a rede se comportaria como um modelo linear \cite{badiger2022retrospective}. Cada camada da rede pode ter uma função de ativação diferente, sendo algumas mais adequadas para camadas ocultas e outras para a camada de saída.

A seguir, são apresentadas as três principais funções de ativação:

\subsubsection{ReLU (Rectified Linear Unit)}
Representada por $f(x) = \max(0, x)$, a ReLU é amplamente utilizada em camadas ocultas. Sua principal vantagem é a eficiência computacional e a aceleração da convergência do gradiente. Contudo, pode causar o problema dos neurônios mortos (Dying ReLU), quando valores negativos resultam sempre em zero \cite{agarap2018deep}.

\subsubsection{Sigmoid}
Definida por $f(x) = \frac{1}{1 + e^{-x}}$, transforma a entrada em um valor entre 0 e 1, sendo útil para problemas de classificação binária. Apesar de ser diferenciável e fornecer gradientes suaves, sofre com o problema do gradiente pequeno para valores extremos, o que dificulta o aprendizado \cite{langer2020approximating}.

\subsubsection{Softmax}
A função Softmax é dada por $f(x_i) = \frac{e^{x_i}}{\sum_j e^{x_j}}$ e é utilizada na camada de saída para classificação multiclasse. Ela converte os valores em uma distribuição de probabilidades, acentuando a classe de maior valor \cite{gao2017properties}.

\subsubsection{Regras gerais para escolha de funções de ativação}

A escolha da função de ativação depende do tipo de problema e da arquitetura da rede. A seguir, são apresentadas algumas diretrizes gerais:

\begin{itemize}
    \item \textit{Camada de saída:}
    \begin{itemize}
        \item Regressão: Linear
        \item Classificação binária: Sigmoid
        \item Classificação multiclasse: Softmax
        \item Classificação multirrótulo: Sigmoid
    \end{itemize}
    \item \textit{Camadas ocultas:}
    \begin{itemize}
        \item Redes convolucionais: ReLU
        \item Redes recorrentes: Tanh ou Sigmoid
    \end{itemize}
\end{itemize}

\subsection{Funções de Custo}

As funções de custo são responsáveis por medir o quão distante a saída prevista está da saída real. Elas orientam o processo de treinamento ajustando os pesos da rede para minimizar esse erro \cite{rashid2020survey}.

\subsubsection{Erro Médio Quadrático (MSE)}

O Erro Médio Quadrático (MSE) é uma das funções mais comuns em regressão, penalizando fortemente grandes erros e sendo sensível a outliers \cite{chicco2021advantages}. Ele é definido como a Equação~\ref{eq:mse}:

\begin{equation}
    MSE = \frac{1}{n} \sum_{i=1}^{n} (y_i - \hat{y}_i)^2
    \label{eq:mse}
\end{equation}

\subsubsection{Erro Médio Absoluto (MAE)}

O Erro Médio Absoluto (MAE) é mais robusto a outliers que o MSE, mas pode ser mais difícil de otimizar. Sua fórmula é apresentada na Equação~\ref{eq:mae}:

\begin{equation}
    MAE = \frac{1}{n} \sum_{i=1}^{n} |y_i - \hat{y}_i|
    \label{eq:mae}
\end{equation}

\subsubsection{Função Huber}

A função Huber combina as vantagens do MSE e MAE, sendo menos sensível a outliers e mais estável para pequenos erros \cite{huber1964robust}. Ela é definida pela Equação~\ref{eq:huber}:

\begin{equation}
    L_{\text{Huber}} =
    \begin{cases}
        \frac{1}{2}(y - \hat{y})^2 & \text{se } |y - \hat{y}| \leq \delta \\
        \delta (|y - \hat{y}| - \frac{1}{2} \delta) & \text{caso contrário}
    \end{cases}
    \label{eq:huber}
\end{equation}

\subsubsection{Entropia Cruzada Binária}

A Entropia Cruzada Binária é indicada para problemas de classificação binária \cite{zhang2018cross}, sendo expressa pela Equação~\ref{eq:binary_crossentropy}:

\begin{equation}
    L = -\frac{1}{n} \sum_{i=1}^{n} \left[y_i \log(\hat{y}_i) + (1 - y_i) \log(1 - \hat{y}_i)\right]
    \label{eq:binary_crossentropy}
\end{equation}

\subsubsection{Entropia Cruzada Categórica}

Já a Entropia Cruzada Categórica é utilizada em classificação multiclasse com rótulo único por amostra, e sua fórmula é apresentada na Equação~\ref{eq:categorical_crossentropy}:

\begin{equation}
    L = -\sum_{i=1}^{n} \sum_{j=1}^{k} y_{ij} \log(\hat{y}_{ij})
    \label{eq:categorical_crossentropy}
\end{equation}

\subsection{Otimizadores}

Otimizadores são algoritmos que ajustam os pesos da rede neural com base no gradiente da função de custo. Eles influenciam diretamente a velocidade e a qualidade da convergência \cite{ruder2016overview}. Entre os principais otimizadores, destaca-se o Gradiente Descendente, que é a forma mais simples de otimização, mas pode ser lenta e sensível à escolha da taxa de aprendizado. O Gradiente Descendente Estocástico (SGD) atualiza os pesos a cada amostra, adicionando ruído estocástico que pode ajudar a escapar de mínimos locais. O Momentum acrescenta uma fração do gradiente anterior ao atual, acelerando a convergência e suavizando oscilações. O RMSProp ajusta a taxa de aprendizado para cada parâmetro com base na média móvel dos gradientes quadrados \cite{tieleman2012lecture}. Por fim, o Adam combina Momentum e RMSProp, sendo amplamente utilizado por sua eficiência e robustez \cite{kingma2014adam}.

\section{Datasets}

% Definindo datasets e sua importância
Os conjuntos de dados são fundamentais para o treinamento de redes neurais, fornecendo as amostras necessárias para aprender padrões complexos e realizar previsões precisas. A qualidade, quantidade e diversidade dos dados impactam diretamente o desempenho dos modelos, especialmente em aprendizagem profunda, onde redes com milhões de parâmetros requerem grandes volumes de dados anotados. Conjuntos como o ImageNet, com mais de 14 milhões de imagens em milhares de categorias, foram essenciais para avanços em visão computacional, como a classificação de imagens e detecção de objetos \cite{deng2009imagenet}. Na engenharia elétrica, especificamente na detecção de falhas em cadeias de isoladores e equipamentos de linhas de transmissão, datasets são frequentemente escassos ou desbalanceados, limitando a capacidade dos modelos de generalizar \cite{shorten2019survey}.

% Datasets escassos: definição e desafios
Datasets escassos, caracterizados por um número reduzido de amostras, são comuns em áreas onde a coleta de dados é custosa, demorada ou restrita pela raridade de eventos, como na detecção de falhas em isoladores de linhas de transmissão. A escassez de dados aumenta o risco de sobreajuste, onde modelos de redes neurais memorizam os dados de treinamento em vez de aprender padrões generalizáveis \cite{goodfellow2016deep}. Esse problema é agravado em aplicações de engenharia elétrica, onde imagens de falhas, como quebras ou flashovers por poluição, são difíceis de obter em quantidade suficiente. A anotação manual de imagens capturadas por drones, frequentemente usada para identificar defeitos, é demorada e propensa a erros, limitando ainda mais o tamanho dos datasets \cite{zheng2022improved}. Por exemplo, o conjunto de dados Insulator-Defect Detection, com apenas 1600 imagens, enfrenta desafios devido ao número limitado de amostras de defeitos \cite{zheng2022improved}.

% Aprendizado por transferência para datasets escassos
O aprendizado por transferência é uma técnica amplamente utilizada para mitigar os desafios de datasets escassos, permitindo que modelos pré-treinados em grandes conjuntos de dados, como o ImageNet, sejam ajustados para tarefas específicas com menos dados \cite{pan2010survey}. Na detecção de falhas em isoladores, \citeonline{zheng2022improved} utilizaram um modelo YOLOv7 pré-treinado, ajustado para detectar defeitos em isoladores a partir de imagens capturadas por veículos aéreos não tripulados (UAVs). Ajustando o modelo para o conjunto de dados Insulator-Defect Detection, os autores alcançaram alta precisão com um número limitado de imagens anotadas. Essa abordagem é eficaz em cenários onde características visuais gerais, como bordas e texturas, podem ser transferidas de datasets genéricos para aplicações especializadas em engenharia elétrica. Outra aplicação foi demonstrada por \citeonline{zou2024bolt}, que utilizaram um modelo YOLOv5 pré-treinado para detectar defeitos em parafusos de linhas de transmissão, aproveitando pesos pré-treinados no conjunto COCO para melhorar a detecção em datasets pequenos.

% Aumento de dados para datasets escassos
O aumento de dados é uma estratégia essencial para lidar com datasets escassos, ampliando artificialmente o tamanho e a diversidade dos dados por meio de transformações como rotação, escala, inversão e alterações de iluminação \cite{shorten2019survey}. Na detecção de falhas em linhas de transmissão, essas transformações ajudam a simular diferentes condições ambientais, como variações de luz ou ângulos de captura, comuns em imagens de UAVs. Por exemplo, \citeonline{peng2023edf} aplicaram aumento de dados para melhorar a robustez de um modelo YOLOv5 na detecção de defeitos pequenos, como flashovers por poluição, em imagens de linhas de transmissão. Métodos avançados, como redes adversárias generativas (GANs), podem gerar imagens sintéticas de falhas, mas sua aplicação em engenharia elétrica é limitada devido à complexidade computacional \cite{goodfellow2014generative}. Apesar disso, GANs têm potencial para criar amostras sintéticas de defeitos raros, como quebras em isoladores, ampliando datasets escassos.

% Aprendizado semi-supervisionado para datasets escassos
O aprendizado semi-supervisionado é uma abordagem promissora para datasets escassos, aproveitando dados não rotulados, que são mais abundantes em inspeções de linhas de transmissão, para melhorar o desempenho do modelo \cite{van2020survey}. Imagens de UAVs capturadas durante inspeções rotineiras podem ser usadas para aprender representações gerais, mesmo sem anotações detalhadas. Embora não haja exemplos específicos na literatura revisada aplicando aprendizado semi-supervisionado diretamente à detecção de falhas em isoladores, \citeonline{chen2020simple} demonstraram sua eficácia em tarefas de visão computacional, sugerindo potencial para aplicações futuras em engenharia elétrica, onde dados não rotulados de inspeções são comuns. Essa técnica pode ser explorada para pré-treinar modelos em grandes conjuntos de imagens de linhas de transmissão antes de ajustá-los em datasets rotulados menores.

% Datasets desbalanceados: definição e desafios
Conjuntos de dados desbalanceados são prevalentes na detecção de falhas em linhas de transmissão, onde amostras de isoladores saudáveis superam significativamente as de defeitos, como quebras ou flashovers por poluição. Esse desbalanceamento pode levar a modelos enviesados que favorecem a classe majoritária, resultando em baixa sensibilidade para a detecção de falhas \cite{he2009learning}. Por exemplo, no conjunto de dados IDID, a proporção de isoladores saudáveis para defeituosos é de aproximadamente 10:1 para quebras e 5:1 para flashovers, o que dificulta a classificação precisa das classes minoritárias \cite{oberweger2024xai}. Esse desafio é crítico em aplicações de engenharia elétrica, onde a detecção de falhas raras é essencial para garantir a segurança e a confiabilidade do sistema.

% Reamostragem para datasets desbalanceados
A reamostragem é uma técnica comum para abordar o desbalanceamento, envolvendo sobreamostragem da classe minoritária ou subamostragem da classe majoritária para equilibrar a distribuição \cite{johnson2019survey}. Métodos como o SMOTE (Synthetic Minority Over-sampling Technique) geram amostras sintéticas da classe minoritária interpolando exemplos existentes \cite{chawla2002smote}. Em \citeonline{oberweger2024xai}, os autores aplicaram subamostragem para criar partições balanceadas do conjunto de dados IDID, retrainando a última camada do modelo com regressão logística para melhorar a classificação de isoladores defeituosos. Essa abordagem aumentou a precisão em até 12\% para quebras e 7\% para flashovers, demonstrando eficácia em cenários desbalanceados. A reamostragem é particularmente útil quando o número de amostras de defeitos é extremamente baixo, como em datasets de inspeção de linhas.

% Perda focal para datasets desbalanceados
A perda focal é uma função de perda especializada que atribui maior peso a exemplos difíceis, frequentemente pertencentes à classe minoritária, reduzindo o impacto de amostras bem classificadas \cite{lin2017focal}. Na detecção de falhas em isoladores, \citeonline{zheng2022improved} utilizaram a perda focal em um modelo YOLOv7 para lidar com o desbalanceamento entre amostras de isoladores e defeitos em imagens de UAVs. A aplicação dessa técnica melhorou a precisão na detecção de defeitos pequenos, como flashovers por poluição, que são menos frequentes. A perda focal é particularmente vantajosa em tarefas de detecção de objetos, onde o fundo da imagem pode dominar a distribuição de classes, como em imagens de linhas de transmissão com fundos complexos.

% Conjuntos de dados unificados
A unificação de datasets públicos, como proposto por \citeonline{felix2020unifying}, oferece uma abordagem valiosa para a pesquisa em detecção de falhas. O repositório combina datasets como o de Tomaszewski et al. e o CPLID, fornecendo imagens e anotações no formato COCO. Essa consolidação facilita o acesso a dados diversificados, embora o desbalanceamento e a escassez permaneçam desafios que requerem técnicas avançadas de pré-processamento e treinamento. A unificação de datasets é particularmente útil para aumentar o número de amostras disponíveis, permitindo treinar modelos mais robustos para detecção de falhas em isoladores \cite{felix2020unifying}.

A Tabela \ref{tab:tecnicas_escassez_desbalanceamento_datasets} resume os principais datasets utilizados na detecção de falhas em isoladores, destacando suas características e desafios.

\begin{table}[H]
\centering
\caption{Técnicas para Lidar com Escassez e Desbalanceamento de Dados}
\label{tab:tecnicas_escassez_desbalanceamento_datasets}
\begin{tabular}{|p{3.5cm}|p{3.5cm}|p{7.5cm}|}
\hline
\textbf{Categoria} & \textbf{Técnica} & \textbf{Descrição} \\
\hline
Escassez & Aprendizado por Transferência & Reutiliza modelos pré-treinados em grandes conjuntos para tarefas com poucos dados \cite{pan2010survey}. \\
\hline
Escassez & Aumento de Dados & Aplica transformações (e.g., rotação, escala) para ampliar o conjunto de dados \cite{shorten2019survey}. \\
\hline
Escassez & Aprendizado Semi-Supervisionado & Aproveita dados não rotulados para aprender representações gerais \cite{van2020survey}. \\
\hline
Desbalanceamento & Reamostragem & Sobreamostra a classe minoritária ou subamostra a majoritária \cite{johnson2019survey}. \\
\hline
Desbalanceamento & Perda Focal & Modifica a perda para focar em exemplos difíceis, geralmente da classe minoritária \cite{lin2017focal}. \\
\hline
\end{tabular}
\end{table}
    
\section{Métricas de Avaliação de Desempenho de Modelos}

Na detecção de falhas em cadeias de isoladores e equipamentos de linhas de transmissão, a escolha das métricas de avaliação é fundamental para garantir a eficácia e confiabilidade dos modelos de redes neurais. Métricas comumente utilizadas incluem acurácia, precisão, recall, F1-score, área sob a curva ROC (AUC-ROC) e, para tarefas de detecção de objetos, a média da precisão média (mAP). A acurácia, definida como a proporção de previsões corretas em relação ao total, pode ser enganosa em datasets desbalanceados, onde a classe de falhas (e.g., quebras ou flashovers) é significativamente menos representada que a classe de isoladores saudáveis. Por exemplo, em um dataset com 95\% de amostras saudáveis, um modelo que sempre prevê "saudável" alcançará alta acurácia, mas falhará em detectar falhas, comprometendo a segurança do sistema \cite{he2009learning}. Em \citeonline{alam2025robust}, os autores reportaram uma acurácia de 99,96\% para um modelo ensemble RF-LSTM Tuned KNN, mas complementaram a avaliação com precisão, recall e F1-score para abordar o desbalanceamento, garantindo uma análise mais robusta do desempenho em classes minoritárias.

A precisão mede a proporção de verdadeiros positivos entre todas as previsões positivas, sendo útil para avaliar a confiabilidade das detecções de falhas. O recall, por outro lado, mede a proporção de verdadeiros positivos entre todas as amostras positivas reais, sendo crítico em aplicações onde falsos negativos (falhas não detectadas) podem levar a falhas catastróficas no sistema de transmissão. O F1-score, a média harmônica entre precisão e recall, oferece uma métrica balanceada que considera ambos os aspectos, sendo amplamente utilizado em cenários desbalanceados \cite{johnson2019survey}.

Para tarefas de detecção de objetos, como identificar defeitos em imagens de UAVs, a métrica mAP é padrão, calculando a precisão média para cada classe e tomando a média geral. Em \citeonline{zheng2022improved}, o modelo YOLOv7 foi avaliado no conjunto Insulator-Defect Detection, utilizando mAP para medir a precisão na detecção de quebras e flashovers, com resultados superiores em comparação com modelos anteriores. A AUC-ROC, que avalia a capacidade do modelo de distinguir entre classes em diferentes limiares, também é relevante, especialmente em classificações binárias (falha vs. não falha). Em \citeonline{alam2025robust}, a AUC-ROC foi reportada como 1,0 para classificações binárias, indicando excelente separação entre classes, mas valores menores (e.g., 0,94 para algumas classes) foram observados em classificações multi-label, refletindo a complexidade de datasets desbalanceados.

Outras métricas, como a matriz de confusão, fornecem uma visão detalhada dos erros do modelo, permitindo calcular taxas de falsos positivos e negativos. Em \citeonline{alam2025robust}, matrizes de confusão foram usadas para avaliar o desempenho do modelo ensemble em classificações binárias e multi-label, complementando as métricas de precisão e recall. Para aplicações em tempo real, como inspeções de UAVs, métricas adicionais, como tempo de inferência e complexidade computacional, também são consideradas, especialmente em modelos como YOLOv5 e YOLOv7, que priorizam eficiência \cite{peng2023edf}. A escolha das métricas deve, portanto, alinhar-se com os objetivos da aplicação, priorizando recall para segurança e mAP para detecção precisa de objetos em imagens.

A Tabela~\ref{tab:metricas_avaliacao_modelos} apresenta um resumo das principais métricas utilizadas para avaliar o desempenho de modelos em tarefas de detecção de falhas em linhas de transmissão, destacando suas características e aplicações.

\begin{table}[H]
\centering
\caption{Principais métricas de avaliação de desempenho de modelos}
\label{tab:metricas_avaliacao_modelos}
\begin{tabular}{|p{4.5cm}|p{10cm}|}
\hline
\textbf{Métrica} & \textbf{Descrição} \\
\hline
Acurácia & Proporção de previsões corretas em relação ao total de amostras. Pode ser enganosa em conjuntos desbalanceados. \\
\hline
Precisão & Proporção de verdadeiros positivos entre todas as previsões positivas. Mede a confiabilidade das detecções. \\
\hline
Recall (Sensibilidade) & Proporção de verdadeiros positivos entre todas as amostras positivas reais. Mede a capacidade de encontrar todos os casos positivos. \\
\hline
F1-score & Média harmônica entre precisão e recall. Útil para avaliar o desempenho em cenários desbalanceados. \\
\hline
mAP (mean Average Precision) & Média das precisões médias para cada classe. Padrão em tarefas de detecção de objetos. \\
\hline
AUC-ROC & Área sob a curva ROC. Mede a capacidade do modelo de distinguir entre classes em diferentes limiares. \\
\hline
Matriz de Confusão & Tabela que mostra as previsões corretas e incorretas, detalhando verdadeiros/falsos positivos e negativos. \\
\hline
\end{tabular}
\end{table}

\chapter{Metodologia}
\section{Definição das Métricas para Avaliar Eficácia dos Processamentos}
% Nesta seção, serão definidas as métricas específicas que serão utilizadas para avaliar a eficácia dos processamentos de imagem no estudo.

O fluxograma apresentado na Figura \ref{fig:fluxograma_metodologia} descreve o processo de desenvolvimento de uma metodologia para o processamento de imagens. Inicialmente, define-se e desenvolve-se os processamentos necessários, ajustando seus parâmetros e combinando-os para otimizar os resultados. Com isso, são avaliadas as métricas de desempenho para verificar a eficácia das abordagens adotadas. Se os resultados forem satisfatórios, os melhores processamentos são selecionados e o processo é finalizado. Caso contrário, analisa-se o número de tentativas realizadas: se muitas tentativas falhas ocorreram, o processamento é descartado e novas abordagens são consideradas. Se o número de tentativas ainda for baixo, o processo retorna à etapa de ajuste de parâmetros, permitindo novas tentativas até atingir os resultados desejados.

\begin{figure}[H]
    \label{fig:fluxograma_metodologia}
    \centering
    \caption{Fluxograma da metodologia de processamento de imagens}
    \includegraphics[width=0.7\textwidth]{img/metodologia.png}
    \fonte{Autor.}
\end{figure}

\section{Escolha do Tipo de Modelo de Rede Neural}
Será discutida a escolha do tipo de modelo de rede neural mais adequado para as tarefas de classificação, detecção e regressão no contexto do estudo.

\section{Seleção dos Datasets para Avaliação}
Serão apresentados os critérios e a seleção dos datasets que serão utilizados para a avaliação dos processamentos de imagem.

\section{Metodologia para Combinação de Processamentos Unitários}
Aqui, será detalhada a metodologia desenvolvida para combinar diferentes processamentos unitários de imagem visando a melhoria dos resultados.

\section{Implementação de um Método de Ajuste Automático de Parâmetros}
Será descrito o método implementado para ajuste automático de parâmetros das técnicas de processamento de imagem, com o objetivo de otimização sem intervenção manual extensa.

\section{Construção de Redes Neurais para Avaliação dos Processamentos}
Nesta seção, será detalhada a construção dos modelos de redes neurais utilizados para avaliar os processamentos de imagem.

\section{Testes com Diferentes Arquiteturas e Análise de Variações nos Resultados}
Serão apresentados os testes realizados com diferentes arquiteturas de redes neurais e a análise das variações nos resultados obtidos.

\chapter{Coleta e Análise de Resultados}
\section{Impacto dos Modelos no Desempenho dos Processamentos}
Será analisado o impacto dos diferentes modelos de redes neurais no desempenho dos processamentos de imagem.

\section{Influência dos Datasets nos Resultados}
Aqui, será discutida a influência dos diferentes datasets nos resultados dos processamentos de imagem.

\section{Comparação entre Diferentes Estratégias de Processamento}
Serão comparadas as diferentes estratégias de processamento de imagem utilizadas no estudo, destacando as vantagens e desvantagens de cada uma.

\chapter{Conclusão}
\section{Síntese dos Resultados Obtidos}
Nesta seção, será feita uma síntese dos principais resultados obtidos ao longo do estudo.

\section{Limitações e Desafios Encontrados}
Serão discutidas as limitações e os desafios encontrados durante a realização do trabalho.

\section{Sugestões para Pesquisas Futuras}
Por fim, serão apresentadas sugestões para pesquisas futuras, com base nos resultados e nas limitações identificadas no estudo.


\bibliographystyle{plain}
\bibliography{referencias}
         
\end{document}
