\chapter{Introdução}
\par Insira aqui a introdução!!!

\lipsum[1-5]



\chapter{Algumas notas sobre bibliografia}


\par O arquivo \textit{referencias.bib} é o nome padrão das referências para este modelo. Para alterá-lo modifique o argumento entre chaves na definição de bibliografia.


   \section{Modelos de citação}
      \par No arquivo modelo de referências, também existem alguns exemplos de diferentes classes de citações. Todas elas podem ser usadas com o \textit{$\backslash$cite$\{$label$\}$} ou \textit{$\backslash$citeonline$\{$label$\}$}, dependendo da forma de citação\footnote{Este é um teste de nota de rodapé}.

	\par Além disso, pode-se incluir obras na bibliografia que nortearam o trabalho mesmo que elas não apareçam diretamente no texto, utilizando o comando \textit{$\backslash$nocite$\{$label$\}$}. Além disso, pode-se citar vários trabalhos em conjunto, por exemplo:
	
	\begin{center}\rule{0.5\textwidth}{1pt}\\$\backslash cite\{label1,label2,label3,...\}$\end{center}

	\begin{verbatim}
	Os ventos do norte não movem moinhos \cite{tcc:mintegui2014, 
	diss:anabor2004, tese:anabor2008, livro:halliday28ed, 
	livro:fedorova:v1, site:amsglo:fog}.
	\end{verbatim}
	    
	\par Os ventos do norte\footnote{Este é um teste de nota de rodapé} não movem moinhos \cite{tcc:mintegui2014,diss:anabor2004,tese:anabor2008,livro:halliday28ed,livro:fedorova:v1,site:amsglo:fog}.
        
        \begin{center}\rule{0.5\textwidth}{1pt}\\$\backslash citeonline\{label1,label2,label3,...\}$\end{center}
        
	\begin{verbatim}
	Segundo \citeonline{cap:livro:djuric1994, livro:aris1989, 
	livro:cotton1989, cap:livro:wyngaard1981, livro:fedorova:v1, 
	artigo:fujita1981,puhalesBLT2010,artigo:janjic2002,
	cbmet:anabor2012,site:wrfhome}, os ventos do norte não movem moinhos.
	\end{verbatim}
            
        \par Segundo \citeonline{cap:livro:djuric1994, livro:aris1989, livro:cotton1989, cap:livro:wyngaard1981, livro:fedorova:v1, artigo:fujita1981,puhalesBLT2010,artigo:janjic2002,cbmet:anabor2012,site:wrfhome}, os ventos\footnote{Este é um teste de nota de rodapé} do norte não movem moinhos.
        \begin{center}\rule{0.5\textwidth}{1pt}\end{center}   
         \par OBS: Trabalhos com mais de três autores aparecerão com a abreviatura ``et al'' no corpo do texto, porém todos os nomes da bibliografia (norma da UFSM que é definida nas opões do abntcite nas definições do arquivo). 
  
  \subsection{O comando \textit{apud}}

  \par A citação \textit{apud} ocorre quando você cita algum autor através de outra obra, sem ter consultado-a propriamente. Neste caso a citação é feita da seguinte forma:
  \begin{center}
  \rule{0.5\textwidth}{1pt}\\
  $\backslash apud\{material\_citado\_no\_material\_lido\}\{material\_lido\}$ \\
  \end{center}
\begin{verbatim}
Sobre a circulação geral da atmosfera pode-se dizer que os ventos do norte
não movem moinhos \apud{apud:richardson1922}{livro:monin:v1}.
\end{verbatim}
  
  Sobre a circulação geral da atmosfera pode-se dizer que os ventos do nortenão movem moinhos \apud{apud:richardson1922}{livro:monin:v1}.
\begin{center}\rule{0.5\textwidth}{1pt}\end{center}  
  \par Nesse caso, na bibliografia só constará a obra consultada e não aquele referenciada pela obra. Para que isso ocorra naturalmente, a obra consultada deve ser incluída normalmente no arquivo referencias.bib enquanto a obra referenciada indiretamente deve ser incluída com a opção \textit{@hidden}, conforme o modelo de referências\footnote{Isto é um teste de nota de rodapé}.

      \subsubsection{\textit{Apud on line}}

      
      \par O \textit{apudonline} se aplica da mesma maneira que o \textit{apud} descrito anteriormente. O termo \textit{on line} é alusivo ao \textit{$\backslash$citeonline$\{$label$\}$} definido no abntex. Nesse caso a citação é feita da seguinte forma:
      \begin{center}
      \rule{0.5\textwidth}{1pt}
            $\backslash apudonline\{material\_citado\_no\_material\_lido\}\{material\_lido\}$ \\
	    \end{center}

 \begin{verbatim}
Segundo \apudonline{apud:richardson1922}{livro:monin:v1}, os ventos do
norte não movem moinhos.
\end{verbatim}

            Segundo \apudonline{apud:richardson1922}{livro:monin:v1}, os ventos do norte não movem moinhos.

\subsubsection{Citação longa}
	
	\begin{quoting}[rightmargin=0cm,leftmargin=4cm]
		\begin{singlespace}
			{\footnotesize
				\noindent Texto da citação sem ser identado e com recuo de 4cm, fonte tamanho 10 e espaçamento simples.  \lipsum[2] Termine com ponto antes da citação.  \cite{man:MDTUFSM2015}.
			}
		\end{singlespace}
	\end{quoting}

       \paragraph{Teste de seção quinária}
       
       \par Texto texto texto.
       
       
         \chapter{Tabelas, figuras, quadros, ilustrações e gráficos}
         
         \par Na MDT da UFSM há uma clara diferença entre tabelas e quadros, quanto a sua apresentação. Aqui, para inserir tabelas usa-se o ambiente tradicionalmente definido \textit{table}. A partir deste modelo simples:
        
         \begin{verbatim}
		\begin{table}[ht]
		\centering
		\caption{Modelo de tabela para MDT-UFSM.}
		\begin{tabular}{ c c c }
		\hline
		Abacate & Banana & Canela \\
		\hline
		21 & 34 & 56 \\
		-3 & 245 & 23 \\
		-25 & -0,57 & 2 \\
                \hline
                 \end{tabular}
		\fonte{Adaptado de \citeonline{livro:halliday28ed}.}
		\end{table}
	  \end{verbatim}
         
         \noindent resulta:
         
         \begin{table}[ht]
         \centering
         \caption{Modelo de tabela para MDT-UFSM.}
         \begin{tabular}{ c c c }
         \hline
         Abacate & Banana & Canela \\
         \hline
         21 & 34 & 56 \\
         -3 & 245 & 23 \\
         -25 & -0,57 & 2 \\
         \hline
         \end{tabular}
          \fonte{Adaptado de \citeonline{livro:halliday28ed}.}
         \end{table}
         
         \par Note que, adicionalmente, foi definido um comando novo: ``fonte''. Ele serve para indicar a fonte da tabela quando necessário, mas também pode ser usado em outros ambientes.
         
         \par Para inserir quadros foi criado um novo ambiente: ``quadro''. O ambiente ``quadro'' deve ser usado de forma semelhante a tabela, como o ambiente tabular. Contudo, neste caso, as linhas verticais e horizontais estão sempre presentes. Um exemplo simples é o seguinte: 
         
         
         \begin{verbatim}
	    \begin{quadro}
   	    \caption{Modelo de quadro para MTD-UFSM.}
	    \centering
	    \begin{tabular}{| c |c |c }
	    \hline
	    Abacate & Banana & Canela \\
	    \hline
	    21 & 34 & 56 \\
	    \hline
	    -3 & 245 & 23 \\
	    \hline
	    -25 & -0,57 & 2 \\
	    \hline
	    \end{tabular}
	    \fonte{Adaptado de \citeonline{livro:halliday28ed}.}
	    \end{quadro}
         \end{verbatim}
         
         \noindent resultando:
         
	    \begin{quadro}
	    \caption{Modelo de quadro para MTD-UFSM.}
	    \centering
	    \begin{tabular}{| c |c |c |}
	    \hline
	    Abacate & Banana & Canela \\
	    \hline
	    21 & 34 & 56 \\
	    \hline
	    -3 & 245 & 23 \\
	    \hline
	    -25 & -0,57 & 2 \\
	    \hline
	    \end{tabular}
    	    \fonte{Adaptado de \citeonline{livro:halliday28ed}.}
	    \end{quadro}
	    
	    
         \noindent Assim como para as tabelas, já está definida uma lista de quadros. Além disso, o comando ``fonte'' também pode ser usado aqui se necessário. Vale lembrar que, na MDT-UFSM, as legendas para figuras, tabelas, quadros, ilustrações e gráficos devem ser inseridas no topo da(o) mesma(o). A fonte sempre embaixo.
         
         \par As figuras devem ser inseridas com o ambiente padrão: \textit{figure}. Veja um exemplo simples:
         
         \begin{verbatim}
	    \begin{figure}[ht]
		\caption{\label{exepretex}Sequência dos elementros pré-testuais da MDT-UFSM.}
		\centering
		\includegraphics[width=0.6\textwidth]{figuras/pretextuais.png}
		\fonte{Adaptado de \citeonline{man:MDTUFSM2015}.}
	    \end{figure}
         \end{verbatim}
         
         \begin{figure}[ht]
     	    \caption{\label{exepretex}Sequência dos elementros pré-testuais da MDT-UFSM.}
	    \centering
	    \includegraphics[width=0.6\textwidth]{figuras/pretextuais.png}
            \fonte{Adaptado de \citeonline{man:MDTUFSM2015}.}
         \end{figure}
        
	\par Para inserir ilustrações e gráficos, foram criados novos ambientes: ``ilustracao'' e ``grafico''. Estes ambientes são semelhantes ao ambiente ``figure'', porém geram sua própria lista. A seguir, exemplos da utilização nos novos ambientes.
	
	\begin{verbatim}
	    \begin{ilustracao}[ht]
		\caption{\label{exepretex1}Sequência dos elementros pré-testuais da MDT-UFSM}
                \centering
		\includegraphics[width=0.6\textwidth]{figuras/pretextuais.png}
		\fonte{Adaptado de \citeonline{man:MDTUFSM2015}.}
	    \end{ilustracao}
         \end{verbatim}
         
         \begin{ilustracao}[ht]
     	    \caption{\label{exepretex1}Sequência dos elementros pré-testuais da MDT-UFSM.}
	    \centering
	    \includegraphics[width=0.6\textwidth]{figuras/pretextuais.png}
            \fonte{Adaptado de \citeonline{man:MDTUFSM2015}.}
         \end{ilustracao}
	
         
	\begin{verbatim}
	    \begin{grafico}[ht]
		\centering
		\includegraphics[width=0.6\textwidth]{figuras/estrutura_com.pdf}
		\caption{\label{exepretex3} Um exemplo de utilização do ambiente ``grafico''.}
		\fonte{Próprio autor.}
	    \end{grafico}
         \end{verbatim}
         
	    \begin{grafico}[ht]
	\caption{\label{exepretex3}Um exemplo de utilização do ambiente ``grafico''.}
             \centering
		\includegraphics[width=0.6\textwidth]{figuras/estrutura_com.pdf}
		\fonte{Próprio autor.}
	    \end{grafico}
         
\chapter{Conclusão}

	\par Conclusão do trabalho.
	\lipsum[1-5]


	
	
% % % % % % % % % % % % % % % % % % % % % % % % % % % % % % % % % % % % % % 
% % % % % % % % % % % % FIM DAS PAGINAS TEXTUAIS % % % % % % % % % % % % % % 
% % % % % % % % % % % % % % % % % % % % % % % % % % % % % % % % % % % % % % 



% % % % % % % % % % % % % % % % % % % % % % % % % % % % % % % % % % % % % % 	
% % % % % % % % % % % % % BIBLIOGRAFIA  % % % % % % % % % % % % % % % % % % 
% % % % % % % % % % % % % % % % % % % % % % % % % % % % % % % % % % % % % % 	

\startbibliography % comando para formatar na MDT UFSM
\bibliography{referencias}

	
% % % % % % % % % % % % % % % % % % % % % % % % % % % % % % % % % % % % % 	
% % % % % % % % % % % % % APENDICES % % % % % % % % % % % % % % % % % % %
% % % % % % % % % % % % % % % % % % % % % % % % % % % % % % % % % % % % % 	
\apendice %%%% TEXTOS A PARIR DESTE PONTO SERAO CONSIDERADOS APENDICES

\chapter{Demonstração de algo}
        \par Algo como apêndice.  
         \lipsum[2-10]

          
% % % % % % % % % % % % % % % % % % % % % % % % % % % % % % % % % % % % % % 	
% % % % % % % % % % % % % % % ANEXOS  % % % % % % % % % % % % % % % % % % % 
% % % % % % % % % % % % % % % % % % % % % % % % % % % % % % % % % % % % % % 	
        \anexo    %%%% TEXTOS A PARIR DESTE PONTO SERAO CONSIDERADOS ANEXOS
        
\chapter{Algo interessante que alguém fez}
         \par Algo como anexo.
         \lipsum[2-10]
                  
        \begin{grafico}[ht]
     	    \caption{\label{exepretex2}Orientações para a lombada do trabalho.}
	    \centering
	    \includegraphics[width=0.6\textwidth]{figuras/lombada.png}
            \fonte{Adaptado de \citeonline{man:MDTUFSM2015}.}
         \end{grafico}         
         
         \lipsum[2-10]